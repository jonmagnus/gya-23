\documentclass[norks, draft]{article}

\usepackage{notes}

\bibliography{bibliography.bib}

\author{Jon-Magnus Rosenblad}
\title{Intro til Kryptografi}

\includeonly{%
    sections/set-theory,
    sections/group-theory,
    sections/cosets,
    sections/finite-abelian-groups,
}

\begin{document}

\maketitle

\tableofcontents
\listoftodos

%\section{What is cryptography?}

\begin{displayquote}
    Cryptography [...] is the
\end{displayquote}


\include{sections/set-theory}
\section{Grunnleggende gruppeteori}

\begin{definition}
    En \textit{gruppe} $(G,\ast)$ er en mengde $G$ sammen med en avbildning
    $\ast\colon G\times G\to G$.
    For to elementer $g, h\in G$ skriver vi $\ast(g,h)$ som $g\ast h$.
    Paret $(G, \ast)$ må tilfredsstille at
    \begin{itemize}
        \item det finnes et element $e\in G$ hvor for alle $g\in G$
            har vi $e\ast g = g\ast e = g$.
            Dette elementet er unikt med denne egenskapen (vis dette!)
            og kalles \textit{identitetselementet} $e_G = e \in G$.
        \item for $g, h, i\in G$ har vi
            $g\ast (h\ast i) = (g\ast h)\ast i$,
            altså $\ast$ er \textit{assosiativ}, og
        \item for hver $g$ finnes et element $h\in G$
            slik at $g \ast h = e$.
            Dette elementet $h$ er unikt med denne egenskapen (vis dette!)
            og vi benevner det $h = g^{-1}$, \textit{inverselementet}.
    \end{itemize}
    Vi sier at $\ast$ er \textit{gruppeoperatoren} til gruppen $(G,\ast)$.
\end{definition}

For å gjøre notasjonen enklere skriver vi ofte bare $G$ for gruppen $(G,\ast)$
når gruppeoperatoren er åpenbar.
Vi forkorter ofte også $g\ast h$ som $gh$.

\begin{definition}
    En avbildning $f\colon G\to H$ mellom to grupper
    er en \textit{morfi} om
    \begin{itemize}
        \item $f(e_G) = e_H$, og
        \item for alle $g, g^\prime\in G$ har vi $f(gg^\prime) = f(g)f(g^\prime)$.
    \end{itemize}
\end{definition}

\begin{definition}
    En delmengde $H\subset G$ av en gruppe $(G, \ast)$ er en \textit{undergruppe}
    om
    \begin{itemize}
        \item $e\in H$,
        \item for alle $h, h^\prime \in H$ er $h\ast h^\prime\in H$,
            og
        \item for alle $h\in H$ er $h^{-1}\in H$.
    \end{itemize}
\end{definition}

\begin{remark}
    Inklusjonen av en undergruppe $H\subset G$
    \[
        H\hookrightarrow G
    \]
    er en morfi.
\end{remark}

\begin{example}
    Vi kjenner allerede til mange eksempler på grupper,
    slik som kjeden av delmengder
    \[
        \mathbb Z
        \subset \mathbb Q
        \subset \mathbb R
        \subset \mathbb C
    \]
    hvor vi lar $\mathbb C = (\mathbb C, +)$ være en gruppe under addisjon
    danner en kjede av undergrupper.
    Merk at $\mathbb N$ ikke er en gruppe under addisjon siden det ikke finnes
    invers elementer.

    Om vi fjerner $0$ fra alle mengdene
    \[
        \mathbb Q^\times
        \subset \mathbb R^\times
        \subset \mathbb C^\times
    \]
    og utruster $\mathbb C$ med multiplikasjon isteden,
    så danner dette en kjede med undergrupper.
\end{example}

\begin{example}
    Mengden bestående av ett element $G = \{0\}$ er en gruppe under
    den eneste mulige avbildningen $G\times G\to G$.
    Denne kalles den \textit{trivielle gruppen} og benevnes ofte bare som $0 = G$.
    Akkurat som vi alltid har en inklusjon av den tomme mengden i en mengde
    $\emptyset \subset S$,
    har alltid en inklusjon av den trivielle gruppen $0\hookrightarrow H$
    til en annen $H$ ved å sende $0\mapsto e_H$.

    Merk at den tomme mengden $\emptyset$ ikke er en gruppe
    siden den ikke har et identitetelement.
\end{example}

\begin{example}
    Om $(G, \ast_G), (H, \ast_H)$ er to grupper så danner $(G\times H, \ast)$
    en gruppe hvor $\ast$ er operatoren
    \[
        (g, h)\ast (g^\prime, h^\prime)
        = (g \ast_G g^\prime, h\ast_H h^\prime).
    \]
    Til eksempel er det reelle planet $\mathbb R^2$ en gruppe under vektor-addisjon.
\end{example}

\begin{example}
    La $\mathrm{Bij}(A)$ være mengden av bijeksjoner $A\to A$.
    Da er $(\mathrm{Bij}(A), \circ)$ en gruppe.
\end{example}

\begin{example}
    Mengden $\mathrm{Mat}_{2\times 2}$ av $(2\times 2)$-matriser under addisjon
    danner en gruppe.
\end{example}

\begin{example}
    Mengden $\mathrm{GL}_2(\mathbb R)$ av inverterbare $(2\times2)$-matriser
    med koeffisienter i $\mathbb R$
    danner en gruppe under matrise multiplikasjon.
\end{example}

\begin{example}
    La $f\colon G\to H$ være en morfi.
    Husk at for en avbildning definerte vi bildet av avbildningen
    $\im f\subset H$ som en delmengde.
    Denne delmengden er også en undergruppe.
\end{example}

La $f\colon G\to H$ være en morfi.
I tillegg til bildet $\im f\subset H$ kan vi lage en annen undergruppe.
\begin{definition}
    Vi definerer \textit{kjernen} til $f\colon G\to H$
    som
    \[
        \ker f = \{ g\in G\mid f(g) = e_H\} \subset G.
    \]
\end{definition}

\begin{lemma}
    Kjernen $\ker (f\colon G\to H)\subset G$ av en morfi er en undergruppe av $G$.
\end{lemma}
\begin{proof}
    Vi trenger å vise at $e_G\in \ker f$, at $\ker f$ er lukket under
    gruppe-operatoren, og at det finnes inverser.
    Den førstnevnte følger fra at $f(e_G) = e_H$ siden $f$ er en morfi.

    La $g,g^\prime \in\ker f$, da har vi $f(g) = f(g^\prime) = e_H$,
    men $f(gg^\prime) = f(g)f(g^\prime) = e_H$, så $gg^\prime\in \ker f)$.

    La $g\in \ker f$, så $f(g) = e_H$, men $f(g^{-1}) = {f(g)}^{-1} = e_H$,
    så $g^{-1}\in\ker f$.
\end{proof}

\begin{corollary}
    La $f\colon G\to H$ være en morfi slik at $\ker f = \{e_G\}$
    (skriver ofte $\ker f = 0$).
    Da er $f$ injektiv.
\end{corollary}
\begin{proof}
    Anta det finnes $g,g^\prime$ slik at $f(g) = f(g^\prime)$.
    Da har vi at
    \[
        f(g^{-1}g^\prime) = {f(g)}^{-1}f(g^\prime) = e_H,
    \]
    så $g^{-1}g^\prime\in \ker f$,
    men $\ker f = \{e_G\}$, så $g = g^\prime$.
\end{proof}

Vi så med mengder at om $f\colon A\to B$ er en bijeksjon så finnes
det en avbildning $g\colon B\to A$ slik at $g\circ f = \id_A$
og $f\circ g = \id_B$.
Det tilsvarende konseptet for grupper er ``isomorfier'',
og vi tar denne egenskapen som definisjonen.
\begin{definition}
    En morfi $f\colon G\to H$ er en \textit{isomorfi}
    dersom det finnes en morfi $g\colon H\to G$ slik at
    $g\circ f = \id_G$ og $f\circ g = \id_H$.
\end{definition}

\begin{definition}
    La $(G, \ast_G), (H, \ast_H)$ være to grupper.
    Vi kan definere en operator $\ast_{G\times H}$
    på $G\times H$ ved
    \[
        (g, h)\ast_{G\times H} (g^\prime, h^\prime)
        = (g \ast_G g^\prime, h \ast_H h^\prime).
    \]
    Dette gjør $G\times H$ til en gruppe med identitetselement $(e_G, e_H)$.
\end{definition}

Som med mengder har vi fortsatt de to projeksjonene
\[\begin{tikzcd}
    G
    \rar[<-]{p}
    &
    G\times H
    \rar{q}
    &
    H,
\end{tikzcd}\]
men vi har også inklusjoner den andre veien
\[\begin{tikzcd}
    G
    \rar{\id_G\times e_H}
    &
    G\times H
    \rar[<-]{e_G\times \id_H}
    &
    H,
\end{tikzcd}\]
gitt ved $(\id_G\times e_H)\colon g\mapsto (g, e_H)$
og $(e_G\times \id_H)\colon h\mapsto (e_G, h)$.
Disse tilfredsstiller $p\circ (\id_G\times e_H) = \id_G$
og $q\circ (e_G\times \id_H) = \id_H$.
Merk også at $\im (\id_G\times e_H) = \ker q$.

\subsubsection*{Oppgaver}
\begin{enumerate}
    \item Finn enhetselementet og inverselementene til alle eksemplene på
        grupper nevnt ovenfor.
    \item Vis at det bare finnes ett element $e\in G$ slik
        at $e \ast g = g\ast e = g$ for alle $g\in G$,
        altså at identitetselementet er unikt.
    \item Vis at for enhver $g\in G$ så finnes det bare ett element
        $h\in G$ slik at $gh = e$.
    \item Vis at for en morfi $f\colon G\to H$
        så har vi for alle $g\in G$ at $f(g^{-1}) = {f(g)}^{-1}$.
    \item Vis at for en morfi $f\colon G\to H$ så er $\im f\subset H$ en undergruppe.
    \item
        \begin{enumerate}
            \item La $f\colon G\to H$ være en bijektiv morfi.
                Vis at den bijektive inversen $f^{-1}\colon H\to G$
                er en morfi, altså er $f$ er isomorfi.
        \end{enumerate}
        En morfi $f\colon G\to H$ kalles en \textit{monomorfi}
        om det finnes en morfi $g\colon H\to G$ slik at $g\circ f=\id_G$.
        En morfi kalles en \textit{epimorfi} om det finnes en morfi
        $h\colon H\to G$ slik at $f\circ h = id_H$.
        \begin{enumerate}[resume]
            \item Vis at $f\colon G\to H$ er en monomorfi hvis og bare hvis $f$
                er en injektiv morfi.
            \item Vis at $f\colon G\to H$ er en epimorfi hvis og bare hvis $f$
                er en surjektiv morfi.
        \end{enumerate}
    \item La $\phi\colon G\to G^\prime$ og $\psi\colon H\to H^\prime$
        være to isomorfier.
        Vis at morfien
        \[
            \phi\times\psi\colon\begin{cases}
                G\times H\to G^\prime\times H^\prime
                \\
                (g,h)\mapsto(\phi(g),\psi(h))
            \end{cases}
        \]
        er en isomorfi.
\end{enumerate}
Så langt har vi sett mange eksempler på uendelige grupper,
slik som $\mathbb C$ og dens undergrupper.
Vi ønsker å se på flere eksempler av \textit{endelige} grupper,
det vil si grupper hvor den underliggende mengden er endelig.
\begin{enumerate}[resume]
    \item La $S^1\subset \mathbb C$ være enhetsirkelen
        \[
            S^1 = \{ z\in \mathbb C\mid |z| = 1\}.
        \]
        \begin{enumerate}
            \item Vis at $S^1$ er en gruppe under (kompleks) multiplikasjon.
            \item La $n$ være et heltall og la $\mu_n\subset S^1$ være mengden av $n$-te
                enhetrøtter
                \[
                    \mu_n = \{
                        e^{\frac k n 2\pi i}
                        \mid k\in \mathbb Z
                    \}.
                \]
                Vis at $\# \mu_n = n$ og at $\mu_n$ er en delgruppe av $S^1$.
            \item Ta for deg $\mu_3, \mu_4$ og $\mu_{12}$.
                Vis at $f\colon \mu_{12}\to \mu_3\times \mu_4$
                gitt ved
                \[
                    e^{\frac k {12} 2\pi i}
                    \mapsto (e^{\frac k 3 2\pi i}, e^{\frac k 4 2\pi i})
                \]
                er en isomorfi.

                Det finnes en slik isomorfi $\mu_{nm}\to \mu_n\times \mu_m$
                så lenge $\mathrm{gcd}(n,m) = 1$.
            \item Vis at det ikke finnes noen isomorfi $\mu_4\to \mu_2\times\mu_2$.
            \item La $n, m$ være to heltall slik at $m | n$.
                Vis at $\mu_m\subset \mu_n$ er en undergruppe.
        \end{enumerate}
    \item Ta for deg gruppen $\mathrm{Bij}(S)$ av bijeksjoner $S\to S$
        hvor $S$ er en endelig mengde.
        Vi så at om vi har en bijeksjon $f\colon S\to S^\prime$
        så er avbildningen
        $\tilde f\colon \mathrm{Bij}(S)\to \mathrm{Bij}(S^\prime)$
        gitt ved $\sigma\mapsto f\circ \sigma \circ f^{-1}$
        for hver $(\sigma\colon S\to S)\in\mathrm{Bij}(S)$
        en bijeksjon.
        \begin{enumerate}
            \item Vis at $\tilde f$ er en morfi.
                Det følger dermed at at $\tilde f$ er en isomorfi.
            \item Vis (ved induksjon) at om $S, S^\prime$ er to endelige mengder
                slik at $\# S = \# S^\prime$.
                Da finnes det en bijeksjon $S\to S^\prime$.
        \end{enumerate}
        Vi har nå sett at om to endelige mengder $S, S^\prime$
        har samme antall elementer så er gruppene av bijeksjoner
        $\mathrm{Bij}(S), \mathrm{Bij}(S^\prime)$ isomorfe.
        La $n = \# S$.
        Vi gir denne gruppen det generelle navnet
        \textit{gruppen av permutasjoner av $n$ elementer}
        $S_n = \mathrm{Bij}(\{1,\dots,n\})$.
        \begin{enumerate}[resume]
            \item Vis at $\# S_n = n!$.
        \end{enumerate}
    \item La $G$ være en endelig gruppe.
        En nært beslektet mengde til $S_n$ er mengden
        $\mathrm{Aut}(G) = \mathrm{Iso}(G, G)$
        av isomorfier $G\to G$, såkalte \textit{automorfier} av $G$.
        \begin{enumerate}
            \item Vis at $(\mathrm{Aut}(G), \circ)$ er en gruppe.
            \item Vis at om en bijeksjon $f\colon G\to H$ er en isomorfi,
                så er $\tilde f\colon \mathrm{Aut}(G)\to \mathrm{Aut}(H)$
                en isomorfi.
            \item Vis at $\mathrm{Aut}(\mu_4)$ ikke er isomorf
                med $\mathrm{Aut}(\mu_2\times\mu_2)$.
                [Merk at $\#\mathrm{Aut}(\mu_4) = 2$.]
        \end{enumerate}
\end{enumerate}



\section{Restklasser}

\begin{definition}
    La $H\subset G$ være en undergruppe.
    En \textit{restklasse} av $H$ i $G$ er en mengde på formen
    \[
        gH = \{gh\mid h\in H\}
    \]
    for en $g\in H$.
    Merk at $g\in gH$ siden $e\in H$.
\end{definition}

\begin{example}
    La $G = \mathbb Z$ være gruppen av heltall under addisjon.
    Ta for deg undergruppen $H = 2\mathbb Z = \{ 2n\mid n\in \mathbb Z\}$
    av partall.
    Partallene har to restklasser i $\mathbb Z$,
    nemlig mengden vi får når vi legger til et partall $2m$
    \[
        2m + (2H) = \{2m + 2n = 2(m + n)\mid n\in \mathbb Z\} = 2\mathbb Z
    \]
    som blir partallene selv,
    og mengden vi får når vi legger til et oddetall
    \[
        (2m + 1) + (2H) = \{2m + 1 + 2n
        = 2(m + n) + 1\mid n\in \mathbb Z\} = 1 + 2\mathbb Z
    \]
    som blir alle oddetallene.
\end{example}

\begin{lemma}
    La $g, g^\prime$ være to elementer i $G$,
    og la $H\subset G$ være en undergruppe av $G$.
    Vi har
    \[
        gH = g^\prime H
    \]
    hvis og bare hvis $g^{-1} g^\prime\in H$.
\end{lemma}
\begin{proof}
    Anta at $g H = g^\prime H$,
    så for alle $h\in H$ finnes en $h^\prime\in H$
    slik at $gh = g^\prime h^\prime$,
    men da kan vi regne
    \[\begin{aligned}
        gh &= g^\prime h^\prime \\
        g^{-1} gh = h
           &= g^{-1} g^\prime h^\prime \\
        h {(h^\prime)}^{-1}
           &= g^{-1} g^\prime h^\prime {(h^\prime)}^{-1}
            = g^{-1} g^\prime
    \end{aligned}\]
    så $g^{-1}g^\prime = h {(h^\prime)}^{-1}\in H$ siden $H$ er en undergruppe.

    For den andre veien anta at $g^{-1} g^\prime \in H$.
    Anta for motsigelse at $gH\neq g^\prime H$,
    så ved symmetri kan vi anta (``uten tap av generalitet'')
    at det finnes et element $gh\in gH$
    slik at $gh\notin g^\prime H$.
    Men $g^{-1} g^\prime \in H$,
    så $h^\prime = {(g^{-1} g^\prime)}^{-1} h\in H$,
    så $gh = g (g^{-1}g^\prime) h^\prime = g^\prime h^\prime \in g^\prime H$,
    men vi antok at $gh\notin g^\prime H$, så vi har en motsigelse.
\end{proof}

\begin{corollary}\label{thm:coset-no-partial-intersection}
    Om $gH\cap g^\prime H\neq \emptyset$ så har vi $gH = g^\prime H$.
\end{corollary}

\begin{corollary}
    Restklassene til $H\subset G$ danner en \textit{partisjon} av $G$,
    det vil si vi kan finne en \textbf{delmengde} (ikke en gruppe) elementer
    $S\subset G$ slik at $gH \cap g^\prime H = \emptyset$ for alle $g,g^\prime\in S$
    med $g\neq g^\prime$, og
    \[
        \bigcup_{g\in S} gH = G.
    \]
\end{corollary}
\begin{proof}
    La $\mathscr S$ være mengden delmengder $S\subset G$ slik at
    for alle $g,g^\prime \in G$ så har vi $gH\cap g^\prime = \emptyset$
    for $g\neq g^\prime$.
    Anta $\tilde S\in \mathscr S$ er maksimal, det vil si det finnes ingen
    $S\in \mathscr S$ slik at $\tilde S\subsetneq S$.
    Vi påstår at $\bigcup_{g\in \tilde S} gH = G$.
    Anta for motsigelse at det finnes en $g^\prime\in G$ slik at
    $g^\prime\notin \bigcup_{g\in \tilde S} gH$.
    Da har vi at $g^\prime H\cap gH = \emptyset$ for alle $g\in \tilde S$,
    for ellers har vi $g^\prime \in gH$ ved \cref{thm:coset-no-partial-intersection},
    men da har vi at $\tilde S\cup \{g^\prime\}\in\mathscr S$
    som motsier at $\tilde S$ er maksimal,
    så det finnes ingen slik $g^\prime$ og $\bigcup_{s\in\tilde S} gH = G$.
\end{proof}

\begin{remark}
    Hvordan vet vi egentlig at det i det hele tatt finnes en maksimal
    mengde $\tilde S$ i $\mathscr S$?
    Om gruppen er uendelig kan man tenke seg at vi alltid
    har plass til å legge til flere og flere elementer til $\tilde S$.
    Det som redder oss er at vi vet at $\tilde S$ er ihvertfall mindre enn $G$,
    så vi kan på en litt innviklet måte bruke et ganske abstrakt aksiom
    kalt \textit{Zorns lemma} til å vite at det finnes et maksimalt element,
    men vi vet ikke nødvendigvis hvordan vi skal finne et slikt element!
\end{remark}

\subsection{Kvotientgruppen}

\begin{lemma}
    La $G$ være en gruppe, $H\subset G$ en undergruppe
    og $g\in G$ et element.
    Da er $\# gH = \# H$.
\end{lemma}
\begin{proof}
    Vi har en naturlig avbildning $f\colon H\to gH$ gitt ved $h\mapsto gh$.
    Denne er surjektiv per definisjon av $gH$, så det gjenstår å vise at den er
    injektiv.

    Anta at $f(h) = f(h^\prime)$ for to $h, h^\prime \in H$,
    altså at $gh = gh^\prime$, men da har vi at
    \[
        h
        = g^{-1} f(h)
        = g^{-1} f(h^\prime)
        = h^\prime.
    \]
\end{proof}

\begin{definition}
    La $G$ være en gruppe og $H\subset G$ en undergruppe.
    Vi definerer mengden av restklasser
    \[
        G / H = \{ gH \mid g\in G\}.
    \]
\end{definition}

\begin{remark}
    Merk at avbildningen $G\to G / H$ gitt ved $g\mapsto gH$ ikke er injektiv,
    for det er flere ulike $g\neq g^\prime$ slik at $gH = g^\prime H$,
    og dette er akkurat de parene $(g, g^\prime)$ slik at $g^{-1}g^\prime\in H$.
\end{remark}

\begin{example}
    La $H\subset G$ være endelige grupper.
    Vi har allerede sett at restklassene danner en partisjon av $G$,
    og nå har vi sett at alle restklassene er like store.
    Det følger umiddelbart at vi må ha
    \[
        \# G = \# (G / H) \# H,
    \]
    altså
    \[
        \# G / \# H = \# (G / H)
    \]
    som motiverer notasjonen.
\end{example}

Vi kan tenke oss en naturlig gruppeoperator på mengden $G / H$,
nemlig at for to restklasser $gH$ og $g^\prime H$ definerer vi produktet deres som
\[
    (gH)(g^\prime H) = (gg^\prime)H,
\]
men husk at vi har flere elementer $\hat g\in G$ som gir samme restklasse
$\bar gH = gH$ enda $\bar g\neq g$.
Så for at denne operatoren skal være ``veldefinert'' på $G / H$ trenger
vi at $(\bar gg^\prime) H = (g g^\prime)H$,
det vil si at
${(g g^\prime)}^{-1} (\bar g g^\prime) = {(g^\prime)}^{-1} h g^\prime\in H$
hvor $h = g^{-1} g\in H$,
men dette er ikke automatisk!

\begin{example}
    Ta for deg gruppen av permutasjoner $G = S_3$ av mengden på tre elementer $\{1,2,3\}$,
    og la $H$ være undergruppen bestående av identiteten og permutasjonen
    \[
        (12)\colon \begin{cases}
            1\mapsto 2\\
            2\mapsto 1\\
            3\mapsto 3.
        \end{cases}
    \]
    La $g= (13)$ og $g^\prime (23)$.
    Vi ser at $gH = \{(13)e, (13)(12)\} = \{(13), (123)\}$,
    så $\bar gH = gH$ hvor $\bar g = (123)$,
    men $gg^\prime = (13)(23) = (321)$,
    mens $\bar g g^\prime = (123)(23) = (12)$,
    så
    \[
        (gg^\prime)H = (321)H\neq H = (12)H = (\bar g g^\prime)H.
    \]
\end{example}

\begin{definition}
    En undergruppe $H\subset G$ er \textit{normal}
    dersom
    \[
        gH = Hg = \{hg\mid h\in H\}
    \]
    for alle $g\in G$.
\end{definition}

\begin{lemma}
    Om $H\subset G$ er en normal undergruppe så er ``multiplikasjon''
    veldefinert på mengden av restklasser $G / H$,
    så $G / H$ danner en gruppe -- \textit{kvotientgruppen} av $G$ over $H$.
\end{lemma}
\begin{proof}
    Om vi har $gH = Hg$ for alle $g\in H$ så følger det at
    $H = g^{-1} H g$,
    så $g^{-1}h g\in H$ for alle $h\in H$.
\end{proof}

\begin{lemma}
    La $f\colon G\to H$ være en morfi.
    Da er $\ker f\subset G$ en normal undergruppe.
\end{lemma}
\begin{proof}
    La $vg\in(\ker f) g$.
    Vi ønsker å vise at $vg\in g\ker f$,
    men det er det samme som at $g^{-1} vg\in \ker f$.
    Vi ser at
    \[\begin{aligned}
        f(g^{-1}vg)
        &= f(g^{-1})f(v)
        \\
        &= {(f(g))}^{-1} e_H f(g)
        \\
        &= e_H,
    \end{aligned}\]
    så $g^{-1} vg\in \ker f$.
\end{proof}

\begin{theorem}[Isomorfiteoremet]\label{thm:isomorphism-theorem}
    La $f\colon G\to H$ være en surjektiv morfi.
    Da har vi en isomorfi $\hat f\colon G / \ker f \to H$,
    det vil si vi kan fylle inn følgende avbildning
    \[\begin{tikzcd}
        G
        \rar
        \drar[swap]{f}
        &
        G / \ker f
        \dar[dashed]{\bar f}
        \\
        &
        H.
    \end{tikzcd}\]
\end{theorem}
\begin{proof}
    Vi benevner restklassen $g + \ker f$ i $G / \ker f$
    ved $\bar g$.
    Vi konstruerer en avbildning $\bar f\colon G / \ker f\to H$
    ved at for en $\bar g\in G / \ker f$ setter vi
    $\bar f(\bar g) = f(g)$ for en representant $g$.

    Om vi velger en annen representant $g^\prime$ med
    $\bar g^\prime = \bar g$ har vi at $g^{-1}g^\prime\in\ker f$,
    så $f(g) = f(g)f(g^{-1}g^\prime) = f(gg^{-1})f(g^\prime) = f(g^\prime)$,
    så definisjonen av $\bar f$ er uavhengig av hvordan vi velger representanter.

    For alle $h\in H$ så finnes en $g\in G$ slik at $f(g) = h$,
    så $\bar f(\bar g) = h$, og $\bar f$ er surjektiv.

    La $g, g^\prime$ slik at $f(g) = f(g^\prime)$.
    Da er $f(g^{-1}g^\prime) = f(g)^{-1} f(g^\prime) = e$,
    så $g^{-1}g^\prime\in \ker f$, så $\bar g = \bar g^\prime$.

    Det gjenstår å vise at $\bar f$ er en morfi,
    det vil si at for alle $\bar g, \bar g^\prime \in G / \ker f$
    så er $\bar f(\bar g\bar g^\prime) = \bar f(\bar g)\bar f(\bar g^\prime)$.
    Vi har at $\bar g\bar g^\prime = \overline{(gg^\prime)}$,
    og vi vet allerede at dette er veldefinert siden $\ker f$ er normal,
    så det gjenstår bare å sjekke at $f(g)f(g^\prime) = f(gg^\prime)$
    for et vilkårlig valg av representanter $g, g^\prime$,
    men dette følger av at $f$ er en morfi.
\end{proof}

\subsubsection*{Oppgaver}

\begin{enumerate}
    \item
        \begin{enumerate}
            \item Hvilke restklasser har $3\mathbb Z = \{3n \mid n\in \mathbb Z\}$
                i $\mathbb Z$? Hva med $5\mathbb Z\subset \mathbb Z$?
            \item For et generelt heltall $n\in \mathbb Z$,
                vis at $n\mathbb Z\subset \mathbb Z$ har $n$ restklasser,
                det vil si $\#(\mathbb Z / n\mathbb Z) = n$.
        \end{enumerate}
    \item Bevis \cref{thm:coset-no-partial-intersection}
    \item Vi skal undersøke mengden av restklasser til $\mathbb Z\subset \mathbb R$.
        \begin{enumerate}
            \item Vi har sett at to restklasser $a + \mathbb Z = b + \mathbb Z$
                hvis og bare hvis $(a - b)\in \mathbb Z$.
                Bruk dette til å konstruere en bijeksjon
                $[0,1)\to \mathbb R / \mathbb Z$.
            \item Vi har at $a + \mathbb Z = \mathbb Z + a$,
                så $\mathbb Z\subset \mathbb R$ er en normal undergruppe
                og $\mathbb R / \mathbb Z$ danner en gruppe.
                Vis at avbildningen
                \[
                    \mathrm{exp}\colon \begin{cases}
                        (\mathbb R, +)\mapsto (S^1,\times)\\
                        x\mapsto e^{2\pi i x}
                    \end{cases}
                \]
                er en gruppe-morfi (merk at vi går fra addisjon til multiplikasjon)
                og at den danner en bijeksjon $[0,1)\to S^1$.
                Her er $S^1$ enhetsirkelen
                \[
                    S^1 = \{z\mid |z| = 1\}\subset \mathbb C.
                \]
                Dette viser at vi har en isomorfi $\mathbb R / \mathbb Z \to S^1$.
            \item Kan du forestille deg hva som skjer om vi tar
                $\mathbb R / \mathbb Q$?
        \end{enumerate}
\end{enumerate}


\section{Endelige abelske grupper}

\begin{definition}
    La $G$ være en gruppe.
    En undergruppe $H\subset G$ er \textit{ekte}
    om $H\neq \{e\}$ og $H\neq G$.

    En gruppe $G$ er \textit{simpel} om $G$ ikke har noen ekte undergrupper.
\end{definition}

\begin{theorem}
    Om $A$ er en endelig abelsk gruppe, og $\# A$ er et primtall,
    da er $A$ simpel.
\end{theorem}

\begin{lemma}\label{thm:binomial-identity}
    La $x,y\in \mathbb Z$  og $n \geq 1$ være heltall.
    Da har vi formelen
    \[
        (x + y)^n = \sum_{i = 0}^n \binom n i x^i y^{n - i}.
    \]
\end{lemma}

Av og til tar vi dette som definisjonen av \textit{binomialkoeffisientene}
$\binom n m$,
men om vi heller bruker definisjonen fra Pascals trekant $\binom n 0 = \binom n n = 1$
for alle $n\in\mathbb N$ og $\binom n {m - 1} + \binom n m = \binom {n + 1} m$
for alle $n, m$ kan vi vise \cref{thm:binomial-identity} ved induksjon som følger.
\begin{proof}
    Tilfellet $n = 1$ er enkelt siden
    $(x + y)^1 = x + y$ og $\binom 1 0 = \binom 1 1 = 1$.

    Anta $(x + y)^k = \sum_{i = 0}^k \binom k i x^i y^{k - i}$.
    Vi har
    \[\begin{aligned}
        (x + y)^{k + 1}
        &=  (x + y)(x + y)^k
        \\
        &= (x + y)\sum_{i = 0}^k \binom k i x^i y^{k - i}
        \\
        &= \sum_{i = 0}^k \binom k i x^{i + 1} y^{k - i}
        + \sum_{i = 0}^k \binom k i x^{i} y^{k - i + 1}
        \\
        &= \sum_{i = 1}^{k + 1} \binom k {i - 1} x^{i} y^{(k + 1) - i}
        + \sum_{i = 0}^k \binom k i x^{i} y^{(k + 1) - i}
        \\
        &= \sum_{i = 0}^{k + 1} \binom {k + 1} i x^{i} y^{(k + 1) - i}
    \end{aligned}\]
\end{proof}

\begin{corollary}{Fermats lille teorem}
    La $p$ være et primtall.
    Da har vi at
    \[
        a^p \cong a \mod p
    \]
    for alle $a\in \mathbb Z$.
\end{corollary}
\begin{proof}
    Vi definerer en avbildning $\pi_p\colon \mathbb Z / p\to\mathbb Z / p$
    ved $a\mapsto a^p$.
    La $a,b\in \mathbb Z$.
    Vi ser at
    \[\begin{aligned}
        \pi_p(a + b)
        &= (a + b)^p
        \\
        &= \sum_{i = 0}^p \binom p i a^i b^{p - i}
        \\
        &= a^p + b^p
        \\
        &= \pi_p(a) + \pi_p(b)
    \end{aligned}\]
    siden $p|\binom p i$ for $i\neq 0, p$,
    så $\pi_p$ er en morfi.
    Vi ser også at $\pi_p(1) = 1^p = 1$,
    og det er bare \`en morfi $\mathbb Z / p\to \mathbb Z / p$
    som tilfredsstiller dette, nemlig identiteten $\id_{\mathbb Z / p}$,
    så $\pi_p = \id_{\mathbb Z / p}$.
\end{proof}

\subsubsection*{Oppgaver}



\begin{verbatim}
- Set theory
    - Difference between sets and tuples
    - Maps
        - Injective
        - Surjective
        - Bijective
    - The set of maps of sets is a set
    - The graph of a map
    - ? Russell's paradox
        - Difference between classes and sets
    - Direct products of sets
- Group theory
    - Definition of a group: set with binary operator that has an identity, is associative and has an inverse
    - Group morphism
    - Subgroups
        - The inclusion of a subgroup is a group morphism
    - Kernel and images of group morphism
        - A group with trivial kernel is injective
        - A bijective group morphism is an isomorphism
    - Abelain groups
        - Additive notation
    - Cosets and normal subgroups
    - Quotient group
    - Lagrange's theorem
    - Definition of a simple group
    - All groups of prime order are simple
    - Inductive proof of the binomial identity
    - Fermat's little theorem
    - Isomorphism theorem
    - The kernel is normal
    - Chinese remainder theorem/Clasification of finite abelian groups
\end{verbatim}


\printbibliography

\end{document}

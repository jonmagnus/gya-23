\section{Diffie-Hellman}

Diffie-Hellman er en metode for nøkkelutveksling,
det vil si at to parter Alice og Bob klarer å komme
frem til en felles hemmelig nøkkel uten at en tredje part
kan enkelt beregne hva nøkkelen er.

Algoritmen, etter \cite{diffie_new_1976}, er som følger:
\begin{enumerate}
    \item Alice og Bob er enige om et primtall $p$
        og et tall $2 \leq g \leq p - 2$.
    \item Alice velger et tilfeldig tall $2 \leq a \leq p - 2$
        og beregner $k_a = g^a \mod p$.
        Hun sender denne nøkkelen $k_a$ til Bob.
    \item Bob velger et tilfeldig tall $2 \leq b \leq p - 2$,
        beregner $k_b = g^b\mod p$
        og sender nøkkelen $k_b$ til Alice.
    \item Alice beregner $k_{ab} = k_b^a$
        og Bob beregner $k_{ab}^\prime = k_a^b$.
        Siden $k_b^a = g^{ab} = k_a^b$ har vi
        at $k_{ab} = k_{ab}^\prime$.
        Dette er Alice og Bob sin felles hemmelige nøkkel.
\end{enumerate}

Teorien som gjør at dette fungerer er bare at vi jobber i
${(\mathbb Z / p)}^\times\cong\mathbb Z / (p - 1)$.
En angriper er interessert i å vite $k_{ab}$ gitt den offentlige informasjonen
$k_a, k_b, g$ og $p$.
Tallene $p$ og $g$ er ofte forhåndsbestemt som en del av algoritmen,
så vi kan tenke oss at en angriper er godt forberedt med dette i tankene.
Spesielt vil en angriper kunne vite primtallsfaktoriseringen av $p - 1$.

\subsection{Diskrete logaritmer}\label{sec:discrete-logarithms}

Den tenkte måten å angripe algoritmen er å først finne enten $a$ eller $b$
for å beregne $k_{ab}$.
Generelt er problemet å finne et tall $n$ slik at $g^n \equiv m\mod p$ gitt
$g, m$ og $p$ kjent som diskret logaritme problemet,
ettersom vi tenker oss at dette definerer en logaritme avbildning
$\log_g\colon{(\mathbb Z / p)}^\times\to \mathbb Z / (p - 1)$
slik at $\log_g(m) = n$.
Merk at denne sjeldent gir de samme verdiene som den vanlige reelle logaritmen
$\log\colon \mathbb R^\times\to \mathbb R$.
Merk også at $g$ må være en generator for ${(\mathbb Z / p)}^\times$
for at $\log_g(m)$ skal være definert for alle $m\in{(\mathbb Z / p)}^\times$.

Anta $g$ er en generator for ${(\mathbb Z / p)}^\times$.
Da har vi en isomorfi $\phi\colon \mathbb Z / (p - 1)\to {(\mathbb Z / p)}^\times$
gitt ved $n\mapsto g^n$.
Denne har da en invers
$\phi^{-1}\colon {(\mathbb Z / p)}^\times\to \mathbb Z / (p - 1)$,
som for $g^n$ gir tilbake $n$.
Om vi kan beregne $\phi^{-1}$ har vi altså løst diskret logaritme problemet,
men merk at morfien $\phi^{-1}$ er akkurat avbildningen $\log_g$,
så det å kunne beskrive den multiplikative gruppen ${(\mathbb Z / p)}^\times$
som en additiv gruppe $\mathbb Z / (p - 1)$
er ekvivalent til å løse diskret logaritme problemet for en gitt base.

\subsection{Kinesisk rest}
Selv om vi ikke kan beregne isomorfien
$\log_g\colon {(\mathbb Z / p)}^\times\to \mathbb Z / (p - 1)$,
så kan vi utnytte at det finnes en slik isomorfi.
La $q | p - 1$ være et primtall, og la $d = (p - 1) / q$.
Da kan vi beregne $k_a^d = g^{ad}$.
Merk at ${(k_a^d)}^q = g^{a(p - 1)} = 1$,
så $k_a^d$ kan bare ha \'en av $q$ mulige verdier avhengig
av verdien til $a^\prime = a\mod q$.
Altså kan vi finne verdien av $a^\prime$ ved å løse diskret logaritme problemet
$a^\prime = \log_{g^d}(k_a^d)$ i $\langle g^d\rangle\subset {(\mathbb Z / p)}^\times$.

Det samme holder for høyere potenser av et primtall,
så om vi har en primtallsfaktorisering $q_1^{r_1}\dots q_k^{r_k} = p - 1$
kan vi ved \cref{thm:chinese-remainder} finne $a$ om vi klarer å løse
diskret logaritme problemet for alle delgruppene tilsvarende primtallsfaktorene
i $p - 1$.

Det er derfor viktig for sikkerheten av Diffie-Hellman at $p - 1$ også har store
primtallsfaktorer.

\subsection{Pollard rho}
En standard metode \cite{pollard_monte_1978}
for å løse diskret logaritme problemet baserer seg på at
vi kan dele opp søket etter verdien for $a$.
Vi vet grovt sett at $0\leq a \leq p  - 1$,
så la $q = \lceil \sqrt{p - 1}\rceil$,
det vil si $q$ er det minste heltallet større enn $\sqrt {p - 1}$.
Da finnes heltall $0\leq m < q$ og $0\leq n \leq \lfloor(p - 1) / q\rfloor \leq q$
slik at $a = nq + m$, så $g^{nq + m} = k_a$.
Om vi nå lager en tabell over verdiene for $k_a g^{-m}$
for alle $0\leq m < q$ kan vi gå igjennom verdiene
$g^{nq}$ for alle $0\leq n\leq q$ og se om verdien dukker opp i tabellen.
Når vi får et treff har vi verdier $n$ og $m$ slik at,
$g^{nq} = k_a g^{-m}$,
altså har vi at $g^{nq + m} = k_a$,
så $a = nq + m$.

Fordelen med denne metoden er at vi trenger bare å teste med
$q + (q + 1) \approx 2\sqrt{p - 1}$ tall for $n$ og $m$ sammenlagt,
isteden for å teste alle $p - 1$ mulige verdier for $a$.
Ulempen er at vi trenger å holde styr på en tabell med rundt $q$
verdier, så når $p$ blir dette ofte for mye data å ha kontroll på.

Pollards rho-metode \cite{pollard_monte_1978} deler inn søket på en annen måte.
Her er tanken at vi går ser på verdien av $g^n k_a^m$ for mange verdier av $n$ og $m$
helt til vi finner to par $(n,m)$ og $(n^\prime, m^\prime)$ slik at
\[
    g^n k_a^m = g^{n^\prime} k_a^{m^\prime}.
\]
Dette kan vi skrive om som
\[
    g^{n - n^\prime} = k_a^{m^\prime - m}.
\]
Isåfall kan vi bruke Euklids algoritme til å finne heltall $l, k$
slik at $l (p - 1) + k (m^\prime - m) = d$ hvor $d = \mathrm{gcd}(p - 1, m^\prime - m)$.
Da har vi at
\[\begin{aligned}
    g^{(n - n^\prime)k}
    &= k_a^{(m^\prime - m)k}
    \\
    &= k_a^d k_a^{-l(p - 1)}
    \\
    &= k_a^d
\end{aligned}\]
Siden vi antar det finnes en løsning $g^a = k_a^d$ må vi ha at $d | (n - n^\prime) k$,
så vi kan beregne $a^\prime = (n - n^\prime) k / d$.
Det er ikke gitt at $a^\prime = a$,
siden det finnes flere mulige $d$-te røtter av $g$.
La $\theta = g^{(p - 1) / d}$ benevne den primitive $d$-te roten til $g$
i ${(\mathbb Z / p)}^\times$.
Da finnes \'en $0\leq i < d$ slik at
$g^{a^\prime} \theta^i = k_a$,
så $a^\prime + i\frac {p - 1} d = a$.

Denne metoden har to steg som potensielt dominerer kompleksiteten.
Det første er å finne verdier $(n, m)$ og $(n^\prime, m^\prime)$
slik at $g^n k_a^m = g^{n^\prime} k_a^{m^\prime}$.
Det andre er når vi tester alle mulige verdier for $i$
for å finne riktig rot av $g$.

For sistnevnte er vi reddet av at sjansen for at $d$ skal ha en stor
primtallsfaktor er omvendt proporsjonal med størrelsen på faktoren,
så $d$ har mest sannsynlig små faktorer,
som vi kan løse ved å bruke \cref{thm:chinese-remainder}.

For det førstnevnte steget presenterer Pollard en avbildning (ikke en morfi)
$\rho \colon {(\mathbb Z / p)}^\times {(\mathbb Z / p)}^\times$
som er laget på en slik måte at gitt at gitt eksponentene $n, m$
for $g^n k_a^m$ så kan vi beregne eksponenter $n^\prime, m^\prime$
slik at $g^{n^\prime} k_a^{m^\prime} = \rho(g^n k_a^m)$.

Tanken her er at $\rho$ oppfører seg som en tilfeldig avbildning,
og for en ekte tilfeldig avbildning
$\eta\colon {(\mathbb Z / p)}^\times\to {(\mathbb Z / p)}^\times$
forventer vi at vi får $\eta^n(1) = \eta^{2n}(1)$
for en $n$ i størrelsesorden $\sqrt p$.
Verdien $n$ kalles \textit{epoken} til $\eta$.

Å finne epoken til $\rho$ kan vi gjøre uten å lagre annen informasjon
enn verdien til $\rho^k(1)$ og $\rho^{2k}(1)$
for en gitt $k$ ved å iterere
gjennom alle $k = 1,\dots,n$.
Dette kalles Floyds algoritme.


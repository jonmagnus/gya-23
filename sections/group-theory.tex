\section{Gruppeteori}

\begin{definition}
    En \textit{gruppe} $(G,\ast)$ er en mengde $G$ sammen med en avbildning
    $\ast\colon G\times G\to G$.
    For to elementer $g, h\in G$ skriver vi $\ast(g,h)$ som $g\ast h$.
    Paret $(G, \ast)$ må tilfredsstille at
    \begin{itemize}
        \item det finnes et element $e\in G$ hvor for alle $g\in G$
            har vi $e\ast g = g\ast e = g$.
            Dette elementet er unikt med denne egenskapen (vis dette!)
            og kalles \textit{identitetselementet} $e_G = e \in G$.
        \item for $g, h, i\in G$ har vi
            $g\ast (h\ast i) = (g\ast h)\ast i$,
            altså $\ast$ er \textit{assosiativ}, og
        \item for hver $g$ finnes et element $h\in G$
            slik at $g \ast h = e$.
            Dette elementet $h$ er unikt med denne egenskapen (vis dette!)
            og vi benevner det $h = g^{-1}$, \textit{inverselementet}.
    \end{itemize}
    Vi sier at $\ast$ er \textit{gruppeoperatoren} til gruppen $(G,\ast)$.
\end{definition}

For å gjøre notasjonen enklere skriver vi ofte bare $G$ for gruppen $(G,\ast)$
når gruppeoperatoren er åpenbar.
Vi forkorter ofte også $g\ast h$ som $gh$.

\begin{definition}
    En avbildning $f\colon G\to H$ mellom to grupper
    er en \textit{morfi} om
    \begin{itemize}
        \item $f(e_G) = e_H$, og
        \item for alle $g, g^\prime\in G$ har vi $f(gg^\prime) = f(g)f(g^\prime)$.
    \end{itemize}
\end{definition}

\begin{definition}
    En delmengde $H\subset G$ av en gruppe $(G, \ast)$ er en \textit{undergruppe}
    om
    \begin{itemize}
        \item $e\in H$,
        \item for alle $h, h^\prime \in H$ er $h\ast h^\prime\in H$,
            og
        \item for alle $h\in H$ er $h^{-1}\in H$.
    \end{itemize}
\end{definition}

\begin{remark}
    Inklusjonen av en undergruppe $H\subset G$
    \[
        H\hookrightarrow G
    \]
    er en morfi.
\end{remark}

\begin{example}
    Vi kjenner allerede til mange eksempler på grupper,
    slik som kjeden av delmengder
    \[
        \mathbb Z
        \subset \mathbb Q
        \subset \mathbb R
        \subset \mathbb C
    \]
    hvor vi lar $\mathbb C = (\mathbb C, +)$ være en gruppe under addisjon
    danner en kjede av undergrupper.
    Merk at $\mathbb N$ ikke er en gruppe under addisjon siden det ikke finnes
    invers elementer.

    Om vi fjerner $0$ fra alle mengdene
    \[
        \mathbb Q^\times
        \subset \mathbb R^\times
        \subset \mathbb C^\times
    \]
    og utruster $\mathbb C$ med multiplikasjon isteden,
    så danner dette en kjede med undergrupper.
\end{example}

\begin{example}
    Om $(G, \ast_G), (H, \ast_H)$ er to grupper så danner $(G\times H, \ast)$
    en gruppe hvor $\ast$ er operatoren
    \[
        (g, h)\ast (g^\prime, h^\prime)
        = (g \ast_G g^\prime, h\ast_H h^\prime).
    \]
    Til eksempel er det reelle planet $\mathbb R^2$ en gruppe under vektor-addisjon.
\end{example}

\begin{example}
    La $\mathrm{Bij}(A)$ være mengden av bijeksjoner $A\to A$.
    Da er $(\mathrm{Bij}(A), \circ)$ en gruppe.
\end{example}

\begin{example}
    Mengden $\mathrm{Mat}_{2\times 2}$ av $(2\times 2)$-matriser under addisjon
    danner en gruppe.
\end{example}

\begin{example}
    Mengden $\mathrm{GL}_2(\mathbb R)$ av inverterbare $(2\times2)$-matriser
    med koeffisienter i $\mathbb R$
    danner en gruppe under matrise multiplikasjon.
\end{example}

\begin{example}
    La $f\colon G\to H$ være en morfi.
    Husk at for en avbildning definerte vi bildet av avbildningen
    $\im f\subset H$ som en delmengde.
    Denne delmengden er også en undergruppe.
\end{example}

La $f\colon G\to H$ være en morfi.
I tillegg til bildet $\im f\subset H$ kan vi lage en annen undergruppe.
\begin{definition}
    Vi definerer \textit{kjernen} til $f\colon G\to H$
    som
    \[
        \ker f = \{ g\in G\mid f(g) = e_H\} \subset G.
    \]
\end{definition}

\begin{lemma}
    Kjernen $\ker (f\colon G\to H)\subset G$ av en morfi er en undergruppe av $G$.
\end{lemma}
\begin{proof}
    Vi trenger å vise at $e_G\in \ker f$, at $\ker f$ er lukket under
    gruppe-operatoren, og at det finnes inverser.
    Den førstnevnte følger fra at $f(e_G) = e_H$ siden $f$ er en morfi.

    La $g,g^\prime \in\ker f$, da har vi $f(g) = f(g^\prime) = e_H$,
    men $f(gg^\prime) = f(g)f(g^\prime) = e_H$, så $gg^\prime\in \ker f)$.

    La $g\in \ker f$, så $f(g) = e_H$, men $f(g^{-1}) = {f(g)}^{-1} = e_H$,
    så $g^{-1}\in\ker f$.
\end{proof}

\subsubsection*{Oppgaver}
\begin{enumerate}
    \item Finn enhetselementet og inverselementene til alle eksemplene på
        grupper nevnt ovenfor.
    \item Vis at det bare finnes ett element $e\in G$ slik
        at $e \ast g = g\ast e = g$ for alle $g\in G$,
        altså at identitetselementet er unikt.
    \item Vis at for enhver $g\in G$ så finnes det bare ett element
        $h\in G$ slik at $gh = e$.
    \item Vis at for en morfi $f\colon G\to H$
        så har vi for alle $g\in G$ at $f(g^{-1}) = {f(g)}^{-1}$.
    \item Vis at for en morfi $f\colon G\to H$ så er $\im f\subset H$ en undergruppe.
\end{enumerate}

\subsection{Mer om grupper}

\begin{definition}
    En gruppe $(G, \ast)$ er \textit{abelsk} om
    for alle $g,h\in G$ har vi $g\ast h = h\ast g$.
    En operator som tilfredsstiller dette kalles \textit{kommutativ}.
\end{definition}

\begin{example}
    Alle eksemplene er med addisjon som gruppeoperator ovenfor er abelske.
    Generelt, om vi benevner en operator med symbolet ``$+$'' så er operatoren
    kommutativ.
\end{example}

\begin{example}
    Gruppen av inverterbare $(2\times 2)$-matriser under multiplikasjon
    $\mathrm{GL}_2(\mathbb R)$
    er ikke abelsk, siden til eksempel
    \[
        \begin{bmatrix}
            1 & 1\\
            0 & 1
        \end{bmatrix}
        \begin{bmatrix}
            1 & 0\\
            1 & 1
        \end{bmatrix}
        =
        \begin{bmatrix}
            1 & 1\\
            1 & 2
        \end{bmatrix}
    \]
    mens
    \[
        \begin{bmatrix}
            1 & 0\\
            1 & 1
        \end{bmatrix}
        \begin{bmatrix}
            1 & 1\\
            0 & 1
        \end{bmatrix}
        =
        \begin{bmatrix}
            2 & 1\\
            1 & 1
        \end{bmatrix}.
    \]
\end{example}

\begin{definition}
    La $H\subset G$ være en undergruppe.
    Et \textit{coset} av $H$ i $G$ er en mengde på formen
    \[
        gH = \{gh\mid h\in H\}
    \]
    for en $g\in H$.
    Merk at $g\in gH$ siden $e\in H$.
\end{definition}

\begin{lemma}
    La $g, g^\prime$ være to elementer i $G$,
    og la $H\subset G$ være en undergruppe av $G$.
    Vi har
    \[
        gH = g^\prime H
    \]
    hvis og bare hvis $g^{-1} g^\prime\in H$.
\end{lemma}
\begin{proof}
    Anta at $g H = g^\prime H$,
    så for alle $h\in H$ finnes en $h^\prime\in H$
    slik at $gh = g^\prime h^\prime$,
    men da kan vi regne
    \[\begin{aligned}
        gh &= g^\prime h^\prime \\
        g^{-1} gh = h
           &= g^{-1} g^\prime h^\prime \\
        h {(h^\prime)}^{-1}
           &= g^{-1} g^\prime h^\prime {(h^\prime)}^{-1}
            = g^{-1} g^\prime
    \end{aligned}\]
    så $g^{-1}g^\prime = h {(h^\prime)}^{-1}\in H$ siden $H$ er en undergruppe.

    For den andre veien anta at $g^{-1} g^\prime \in H$.
    Anta for motsigelse at $gH\neq g^\prime H$,
    så ved symmetri kan vi anta (``uten tap av generalitet'')
    at det finnes et element $gh\in gH$
    slik at $gh\notin g^\prime H$.
    Men $g^{-1} g^\prime \in H$,
    så $h^\prime = {(g^{-1} g^\prime)}^{-1} h\in H$,
    så $gh = g (g^{-1}g^\prime) h^\prime = g^\prime h^\prime \in g^\prime H$,
    men vi antok at $gh\notin g^\prime H$, så vi har en motsigelse.
\end{proof}

\begin{corollary}\label{thm:coset-no-partial-intersection}
    Om $gH\cap g^\prime H\neq \emptyset$ så har vi $gH = g^\prime H$.
\end{corollary}

\begin{corollary}
    Cosetene til $H\subset G$ danner en \textit{partisjon} av $G$,
    det vil si vi kan finne en \textbf{delmengde} (ikke en gruppe) elementer
    $S\subset G$ slik at $gH \cap g^\prime H = \emptyset$ for alle $g,g^\prime\in S$
    med $g\neq g^\prime$, og
    \[
        \bigcup_{g\in S} gH = G.
    \]
\end{corollary}
\begin{proof}
    La $\mathscr S$ være mengden delmengder $S\subset G$ slik at
    for alle $g,g^\prime \in G$ så har vi $gH\cap g^\prime = \emptyset$
    for $g\neq g^\prime$.
    Anta $\tilde S\in \mathscr S$ er maksimal, det vil si det finnes ingen
    $S\in \mathscr S$ slik at $\tilde S\subsetneq S$.
    Vi påstår at $\bigcup_{g\in \tilde S} gH = G$.
    Anta for motsigelse at det finnes en $g^\prime\in G$ slik at
    $g^\prime\notin \bigcup_{g\in \tilde S} gH$.
    Da har vi at $g^\prime H\cap gH = \emptyset$ for alle $g\in \tilde S$,
    for ellers har vi $g^\prime \in gH$ ved \cref{thm:coset-no-partial-intersection},
    men da har vi at $\tilde S\cup \{g^\prime\}\in\mathscr S$
    som motsier at $\tilde S$ er maksimal,
    så det finnes ingen slik $g^\prime$ og $\bigcup_{s\in\tilde S} gH = G$.
\end{proof}

\begin{remark}
    Hvordan vet vi egentlig at det i det hele tatt finnes en maksimal
    mengde $\tilde S$ i $\mathscr S$?
    Om gruppen er uendelig kan man tenke seg at vi alltid
    har plass til å legge til flere og flere elementer til $\tilde S$.
    Det som redder oss er at vi vet at $\tilde S$ er ihvertfall mindre enn $G$,
    så vi kan på en litt innviklet måte bruke et ganske abstrakt aksiom
    kalt \textit{Zorns lemma} til å vite at det finnes et maksimalt element,
    men vi vet ikke nødvendigvis hvordan vi skal finne et slikt element!
\end{remark}

\begin{definition}
    En undergruppe $H\subset G$ er \textit{normal}
    dersom
    \[
        gH = Hg = \{hg\mid h\in H\}
    \]
    for alle $g\in G$.
\end{definition}

\begin{lemma}
    La $f\colon G\to H$ være en morfi.
    Da er $\ker f\subset G$ en normal undergruppe.
\end{lemma}
\begin{proof}
    La $vg\in(\ker f) g$.
    Vi ønsker å vise at $vg\in g\ker f$,
    men det er det samme som at $g^{-1} vg\in \ker f$.
    Vi ser at
    \[\begin{aligned}
        f(g^{-1}vg)
        &= f(g^{-1})f(v)
        \\
        &= {(f(g))}^{-1} e_H f(g)
        \\
        &= e_H,
    \end{aligned}\]
    så $g^{-1} vg\in \ker f$.
\end{proof}

\subsubsection*{Oppgaver}

\begin{enumerate}
    \item Bevis \cref{thm:coset-no-partial-intersection}
    \item La $A$ være en abelsk gruppe og $H\subset A$ en undergruppe.
        \begin{enumerate}
            \item Vis at $H$ er abelsk.
            \item Vis at $H$ er en normal undergruppe.
        \end{enumerate}
\end{enumerate}


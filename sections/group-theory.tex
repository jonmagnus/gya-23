\section{Grunnleggende gruppeteori}

\begin{definition}
    En \textit{gruppe} $(G,\ast)$ er en mengde $G$ sammen med en avbildning
    $\ast\colon G\times G\to G$.
    For to elementer $g, h\in G$ skriver vi $\ast(g,h)$ som $g\ast h$.
    Paret $(G, \ast)$ må tilfredsstille at
    \begin{itemize}
        \item det finnes et element $e\in G$ hvor for alle $g\in G$
            har vi $e\ast g = g\ast e = g$.
            Dette elementet er unikt med denne egenskapen (vis dette!)
            og kalles \textit{identitetselementet} $e_G = e \in G$.
        \item for $g, h, i\in G$ har vi
            $g\ast (h\ast i) = (g\ast h)\ast i$,
            altså $\ast$ er \textit{assosiativ}, og
        \item for hver $g$ finnes et element $h\in G$
            slik at $g \ast h = e$.
            Dette elementet $h$ er unikt med denne egenskapen (vis dette!)
            og vi benevner det $h = g^{-1}$, \textit{inverselementet}.
    \end{itemize}
    Vi sier at $\ast$ er \textit{gruppeoperatoren} til gruppen $(G,\ast)$.
\end{definition}

For å gjøre notasjonen enklere skriver vi ofte bare $G$ for gruppen $(G,\ast)$
når gruppeoperatoren er åpenbar.
Vi forkorter ofte også $g\ast h$ som $gh$.

\begin{definition}
    En avbildning $f\colon G\to H$ mellom to grupper
    er en \textit{morfi} om
    \begin{itemize}
        \item $f(e_G) = e_H$, og
        \item for alle $g, g^\prime\in G$ har vi $f(gg^\prime) = f(g)f(g^\prime)$.
    \end{itemize}
\end{definition}

\begin{definition}
    En delmengde $H\subset G$ av en gruppe $(G, \ast)$ er en \textit{undergruppe}
    om
    \begin{itemize}
        \item $e\in H$,
        \item for alle $h, h^\prime \in H$ er $h\ast h^\prime\in H$,
            og
        \item for alle $h\in H$ er $h^{-1}\in H$.
    \end{itemize}
\end{definition}

\begin{remark}
    Inklusjonen av en undergruppe $H\subset G$
    \[
        H\hookrightarrow G
    \]
    er en morfi.
\end{remark}

\begin{example}
    Vi kjenner allerede til mange eksempler på grupper,
    slik som kjeden av delmengder
    \[
        \mathbb Z
        \subset \mathbb Q
        \subset \mathbb R
        \subset \mathbb C
    \]
    hvor vi lar $\mathbb C = (\mathbb C, +)$ være en gruppe under addisjon
    danner en kjede av undergrupper.
    Merk at $\mathbb N$ ikke er en gruppe under addisjon siden det ikke finnes
    invers elementer.

    Om vi fjerner $0$ fra alle mengdene
    \[
        \mathbb Q^\times
        \subset \mathbb R^\times
        \subset \mathbb C^\times
    \]
    og utruster $\mathbb C$ med multiplikasjon isteden,
    så danner dette en kjede med undergrupper.
\end{example}

\begin{example}
    Om $(G, \ast_G), (H, \ast_H)$ er to grupper så danner $(G\times H, \ast)$
    en gruppe hvor $\ast$ er operatoren
    \[
        (g, h)\ast (g^\prime, h^\prime)
        = (g \ast_G g^\prime, h\ast_H h^\prime).
    \]
    Til eksempel er det reelle planet $\mathbb R^2$ en gruppe under vektor-addisjon.
\end{example}

\begin{example}
    La $\mathrm{Bij}(A)$ være mengden av bijeksjoner $A\to A$.
    Da er $(\mathrm{Bij}(A), \circ)$ en gruppe.
\end{example}

\begin{example}
    Mengden $\mathrm{Mat}_{2\times 2}$ av $(2\times 2)$-matriser under addisjon
    danner en gruppe.
\end{example}

\begin{example}
    Mengden $\mathrm{GL}_2(\mathbb R)$ av inverterbare $(2\times2)$-matriser
    med koeffisienter i $\mathbb R$
    danner en gruppe under matrise multiplikasjon.
\end{example}

\begin{example}
    La $f\colon G\to H$ være en morfi.
    Husk at for en avbildning definerte vi bildet av avbildningen
    $\im f\subset H$ som en delmengde.
    Denne delmengden er også en undergruppe.
\end{example}

La $f\colon G\to H$ være en morfi.
I tillegg til bildet $\im f\subset H$ kan vi lage en annen undergruppe.
\begin{definition}
    Vi definerer \textit{kjernen} til $f\colon G\to H$
    som
    \[
        \ker f = \{ g\in G\mid f(g) = e_H\} \subset G.
    \]
\end{definition}

\begin{lemma}
    Kjernen $\ker (f\colon G\to H)\subset G$ av en morfi er en undergruppe av $G$.
\end{lemma}
\begin{proof}
    Vi trenger å vise at $e_G\in \ker f$, at $\ker f$ er lukket under
    gruppe-operatoren, og at det finnes inverser.
    Den førstnevnte følger fra at $f(e_G) = e_H$ siden $f$ er en morfi.

    La $g,g^\prime \in\ker f$, da har vi $f(g) = f(g^\prime) = e_H$,
    men $f(gg^\prime) = f(g)f(g^\prime) = e_H$, så $gg^\prime\in \ker f)$.

    La $g\in \ker f$, så $f(g) = e_H$, men $f(g^{-1}) = {f(g)}^{-1} = e_H$,
    så $g^{-1}\in\ker f$.
\end{proof}

\subsubsection*{Oppgaver}
\begin{enumerate}
    \item Finn enhetselementet og inverselementene til alle eksemplene på
        grupper nevnt ovenfor.
    \item Vis at det bare finnes ett element $e\in G$ slik
        at $e \ast g = g\ast e = g$ for alle $g\in G$,
        altså at identitetselementet er unikt.
    \item Vis at for enhver $g\in G$ så finnes det bare ett element
        $h\in G$ slik at $gh = e$.
    \item Vis at for en morfi $f\colon G\to H$
        så har vi for alle $g\in G$ at $f(g^{-1}) = {f(g)}^{-1}$.
    \item Vis at for en morfi $f\colon G\to H$ så er $\im f\subset H$ en undergruppe.
\end{enumerate}


\section{Gruppeteori}

\subsection{Grunnleggende notasjon og mengdeteori}


\begin{definition}
    En \textit{mengde} er en samling objekter $S$ hvor vi kan for
    ethvert objekt $e$ vite om $e$ tilhører $S$, skrevet $e\in S$,
    eller ikke, skrevet $e\notin S$.

    Objektene som tilhører en mengde kalles \textit{elementene}
    i mengden.
\end{definition}

Vi noterer ofte en mengde med elementene den inneholder,
til eksempel skriver vi de naturlige tallene som
\[
    \mathbb N = \{0, 1, 2, \dots\}.
\]
\begin{example}
    De naturlige tallene $\mathbb N$, heltallene $\mathbb Z$,
    de rasjonale tallene $\mathbb Q$,
    de reelle tallene $\mathbb R$
    og de komplekse tallene $\mathbb C$ danner alle mengder.

    Det finnes en tom mengde $\emptyset = \{\}$,
    og vi kan lage mengder av mengder,
    slik som mengden av den tomme mengden $\{\emptyset\}\neq\emptyset$.
\end{example}

Det finnes også samlinger av objekter som ikke danner mengder.
\begin{example}
    Samlingen av alle mengder som ikke inneholder seg selv er ikke en mengde.
    Om vi snakker om samlinger som ikke er mengder fremhever vi ofte at de ikke
    er mengder ved å bruke for eksempel klammeparenteser,
    så denne samlingen kan skrives som
    \[
        \left[
            S
            \mid
            S\notin S
        \right]
    \]
\end{example}

\begin{remark}
    Et element er enten med i en mengde eller ikke
    -- det finnes ingen id\`e om flere av det samme elementet i en mengde.
    Altså om vi har $S = \{a,b\}$, men $a = b = 1$,
    så vil $S = \{1\}$.

    Om vi vil ha flere elementer, og samtidig bryr oss om rekkefølgen av elementer
    så bruker vi en \textit{tuppel} $(a,b)$,
    så om $a = b = 1$ har vi $(a, b) = (1,1) \neq (1)$,
    men for $a\neq b$ har vi $(a, b)\neq (b, a)$.
\end{remark}

\begin{definition}
    En avbildning $f\colon A\to B$ mellom to mengder er en
    regel som for ethvert element $a\in A$ assosierer et element
    $b = f(a)\in A$.

    Vi tegner ofte opp avbildninger som piler,
    så om vi har flere mengder $A, B, C$, men flere avbildninger
    $f\colon A\to B, g\colon A\to C, h\colon B\to C$
    imellom seg kan vi tegne det opp som
    \[
        \begin{tikzcd}
            A
            \ar{rr}{f}
            \ar{rd}{g}
            &&
            B
            \ar{ld}{h}
            \\
            &
            C.
        \end{tikzcd}
    \]
\end{definition}

For ethvert element $a\in A$ finnes det bare \`en verdi for $f(a)\in B$.
Derimot kan flere verdier $a, a^\prime$ avbilde på samme verdi
$f(a) = f(a^\prime)$.

\begin{example}
    For enhver mengde $A$ finnes det en naturlig avbildning
    $\id_A\colon A\to A$ som sender hvert element $a\in A$
    på seg selv $a\mapsto a$
\end{example}

\begin{remark}
    Merk at for enhver mengde $A$ så finnes det
    nøyaktig en avbildning $\emptyset \to A$,
    men det finnes ingen avbildning $A\to \emptyset$ med mindre $A = \emptyset$.
\end{remark}

\begin{example}
    Om vi har to avbildninger $A\xrightarrow{f} B\xrightarrow{g} C$
    kan vi lage en ny avbildning $g\circ f\colon A\to C$
    ved å sende $g\circ f(a) = g(f(a))$ for alle $a\in A$.
\end{example}

\begin{definition}
    En avbildning $f\colon A\to B$ er \textit{injektiv} om for alle $a,b\in A$ med $a\neq b$
    så er $f(a)\neq f(b)$.
    Om $f$ er injektiv skriver vi ofte pilen som $f\colon A\hookrightarrow B$
    og sier at $f$ er en \textit{inklusjon}.

    Avbildningen er \textit{surjektiv} om for alle $b\in B$ så finnes en $a\in A$
    slik at $f(a) = b$.
    Om $f$ er surjektiv skriver vi noen ganger pilen som
    $f\colon A\twoheadrightarrow B$ og sier at $f$ er en \textit{surjeksjon}.

    Om $f$ er både injektiv og surjektiv sier vi den er \textit{bijektiv}
    og sier $f$ er en \textit{bijeksjon}.
\end{definition}

\begin{lemma}
    Om $f\colon A\to B$ er en bijeksjon så finnes en unik avbildning
    $g\colon B\to A$ slik at $f\circ g = \id_B$ og $g\circ f = \id_A$,
    dvs.
    $f(g(b)) = b$ for alle $b\in B$ og $g(f(a)) = a$ for alle $a\in A$.
    Vi kaller $g$ \textit{inversavbildningen} til $f$ og skriver $f^{-1} = g$.
\end{lemma}
\begin{proof}
    \todo[inline]{Skriv bevis}
\end{proof}

\begin{example}
    For en mengde $A$ er $\id_A$ en bijeksjon,
    og inversavbildningen er avbildningen selv $\id_A = \id_A^{-1}$.
\end{example}

\begin{example}
    Vi kan ta \textit{snitt} og \textit{union} av mengder for å skape nye mengder.
    Snittet av to mengder $A,B$ er mengden av elementer som ligger i både $A$ og $B$
    \[
        A\cap B = \{ e\mid e\in A\mbox{ og } e\in B\},
    \]
    mens unionen er mengden av elementer i enten $A$ eller $B$
    \[
        A\cup B = \{ e\mid e\in A\mbox{ eller } e\in B\}.
    \]

    Når vi tar snitt og union følger det med naturlige inklusjoner
    \[\begin{tikzcd}
        A\cap B
        \rar[hook]
        \dar[hook]
        & B
        \dar[hook]
        \\
        A
        \rar[hook]
        &
        A\cup B
    \end{tikzcd}\].
    Disse er så naturlige at vi benevner dem som $A\cap B\subset A,B\subset A\cup B$
    utenfor diagrammer, altså er snittet $A\cap B$
    en \textit{delmengde} av $A$ (og $B$),
    og $A$ (og $B$) er en delmengde av $A\cup B$.
\end{example}

\begin{example}
    Vi har $\{1,\dots,10\}\cap \{5,\dots,15\} = \{5, 6, 7, 8, 9, 10\}$,
    og $\{1,\dots, 10,\}\cup \{5,\dots, 15\} = \{1,\dots, 15\}$.
\end{example}

\begin{example}
    Vi kan ta produktet av to mengder $A, B$.
    Dette er mengden
    \[
        A\times B = \{
            (a,b)\mid a\in A,\, b\in B
        \}.
    \]
    Her følger også med noen avbildninger
    \[\begin{tikzcd}
        A
        \rar[<-]{p}
        &
        A\times B
        \rar{q}
        &
        B
    \end{tikzcd}\]
    hvor $p\colon (a,b)\mapsto a$ og $q\colon (a,b)\mapsto b$ er
    projeksjonene til første og andre element henholdsvis.
\end{example}

\begin{example}
    Produktmengden til $\{0,\dots,9\}$ med seg selv består av hunder elementer,
    og vi har en bijeksjon $\{0,\dots,9\}\times\{0,\dots,9\}\to \{0,\dots,99\}$
    gitt ved $(n, m)\mapsto 10n + m$.
\end{example}

\begin{definition}
    Vi kan snakke om størrelsen på en mengde $A$ -- altså \textit{kardinaliteten}
    til mengden, som vi benevner $|A|$ eller $\# A$.
    Om $A$ inneholder et endelig antall elementer,
    så er $\# A$ antall elementer i $A$.
    Om $A$ ikke er endelig kan vi fortsatt snakke om kardinaliteten til $A$,
    og vi kan sammenligne kardinaliteter,
    for om det finnes en inklusjon $A\hookrightarrow B$
    kan vi si $\# A\leq \# B$,
    og om det finnes en bijeksjon så har vi $\# A = \# B$.
\end{definition}

\begin{example}
    Vi har
    \[
        \mathbb N
        \subset \mathbb Z
        \subset \mathbb Q
        \subset \mathbb R
        \subset \mathbb C
    \]
    så åpenbart har vi
    \[
        \#\mathbb N
        \leq \#\mathbb Z
        \leq \#\mathbb Q
        \leq \#\mathbb R
        \leq \#\mathbb C,
    \]
    men det viser seg at
    \[
        \#\mathbb N
        = \#\mathbb Z
        = \#\mathbb Q
        < \#\mathbb R
        = \#\mathbb C.
    \]
    Merk at $\mathbb Z$ og $\mathbb R$ har forskjellig kardinalitet,
    men begge er uendelige.
    Vi benevner den første uendeligheten med $\aleph_0 = \#\mathbb Z$,
    og den sistnevnte som $\aleph_1 = \#\mathbb R$.
\end{example}

\begin{example}
    La $A, B$ være to mengder.
    Vi kan lage mengden av alle avbildninger fra $A$ til $B$
    \[
        \mathrm{Map}(A,B) = \{f\colon A\to B\}.
    \]
    Vi sier to avbildninger $f,g\colon A\to B$
    er like om $f(a) = g(a)$ for alle $a\in A$.
\end{example}

\subsection{Oppgaver}

\begin{enumerate}
    \item Vis at om $f\colon A\hookrightarrow B$ er en injeksjon
        så finnes en $g\colon B\twoheadrightarrow A$ slik at
        $g\circ f = \id_A$.
        Er den unik?
    \item La $A\xrightarrow{f} B\xrightarrow{g} C$ være to avbildninger.
        \begin{enumerate}
            \item
                Anta $g\circ f$ er surjektiv.
                Vis at da må $f$ være surjektiv.
            \item
                Anta $g\circ f$ er injektiv.
                Vis at da må $f$ være injektiv.
            \item
                Om $g\circ f$ er en bijeksjon, må $f$ eller $g$ være en bijeksjon?
        \end{enumerate}
    \item La $A, B$ være to endelige mengder, og la $n = \# A$ og $m=\# B$,
        vis at da er $\# \mathrm{Map}(A,B) = m^n$.
        Dette motiverer hvorfor flere forfattere velger å skrive
        $\mathrm{Map}(A,B) = B^A$.
    \item La $\mathrm{Bij}(A)$ benevne mengden av bijeksjoner $s\colon A\to A$,
        og la $f\colon A\to B$ være en bijeksjon.
        Vis at avbildningen $\tilde f\colon \mathrm{Bij}(A)\to \mathrm{Bij}(B)$
        gitt ved $s\mapsto f\circ s\circ f^{-1}$ er en bijeksjon.
\end{enumerate}

\subsection{Gruppeteori}

\begin{definition}
    En \textit{gruppe} $(G,\ast)$ er en mengde $G$ sammen med en avbildning
    $\ast\colon G\times G\to G$.
    For to elementer $g, h\in G$ skriver vi $\ast(g,h)$ som $g\ast h$.
    Paret $(G, \ast)$ må tilfredsstille at
    \begin{itemize}
        \item det finnes et element $e\in G$ hvor for alle $g\in G$
            har vi $e\ast g = g\ast e = g$.
            Dette elementet er unikt med denne egenskapen (vis dette!)
            og kalles \textit{identitetselementet} $e_G = e \in G$.
        \item for $g, h, i\in G$ har vi
            $g\ast (h\ast i) = (g\ast h)\ast i$, og
        \item for hver $g$ finnes et element $h\in G$
            slik at $g \ast h = e$.
            Dette elementet $h$ er unikt med denne egenskapen (vis dette!)
            og vi benevner det $h = g^{-1}$.
    \end{itemize}
    Vi sier at $\ast$ er \textit{gruppeoperatoren} til gruppen $(G,\ast)$.
\end{definition}

For å gjøre notasjonen enklere skriver vi ofte bare $G$ for gruppen $(G,\ast)$
når gruppeoperatoren er åpenbar.
Vi forkorter ofte også $g\ast h$ som $gh$.

\begin{definition}
    En avbildning $f\colon G\to H$ mellom to grupper
    er en \textit{morfi} om
    \begin{itemize}
        \item $f(e_G) = e_H$, og
        \item for alle $g, g^\prime\in G$ har vi $f(gg^\prime) = f(g)f(g^\prime)$.
    \end{itemize}
\end{definition}

\begin{definition}
    En delmengde $H\subset G$ av en gruppe $(G, \ast)$ er en \textit{delgruppe}
    om
    \begin{itemize}
        \item $e\in H$,
        \item for alle $h, h^\prime \in H$ er $h\ast h^\prime\in H$,
            og
        \item for alle $h\in H$ er $h^{-1}\in H$.
    \end{itemize}
\end{definition}

\begin{remark}
    Inklusjonen av en delgruppe
    \[
        H\hookrightarrow G
    \]
    er en morfi.
\end{remark}

\begin{example}
    Vi kjenner allerede til mange eksempler på grupper,
    slik som kjeden av delmengder
    \[
        \mathbb Z
        \subset \mathbb Q
        \subset \mathbb R
        \subset \mathbb C
    \]
    hvor vi lar $\mathbb C = (\mathbb C, +)$ være en gruppe under addisjon
    danner en kjede av delgrupper.
    Merk at $\mathbb N$ ikke er en gruppe under addisjon siden det ikke finnes
    invers elementer.

    Om vi fjerner $0$ fra alle mengdene
    \[
        \mathbb Q^\times
        \subset \mathbb R^\times
        \subset \mathbb C^\times
    \]
    og utruster $\mathbb C$ med multiplikasjon isteden,
    så danner dette en kjede med delgrupper.
\end{example}

\begin{example}
    Om $(G, \ast_G), (H, \ast_H)$ er to grupper så danner $(G\times H, \ast)$
    en gruppe hvor $\ast$ er operatoren
    \[
        (g, h)\ast (g^\prime, h^\prime)
        = (g \ast_G g^\prime, h\ast_H h^\prime).
    \]
    Til eksempel er det reelle planet $\mathbb R^2$ en gruppe under vektor-addisjon.
\end{example}

\begin{example}
    La $\mathrm{Bij}(A)$ være mengden av bijeksjoner $A\to A$.
    Da er $(\mathrm{Bij}(A), \circ)$ en gruppe.
\end{example}

\begin{example}
    Mengden $\mathrm{Mat}_{2\times 2}$ av $(2\times 2)$-matriser under addisjon
    danner en gruppe.
\end{example}

\begin{example}
    Mengden $\mathrm{GL}_2(\mathbb R)$ av inverterbare $(2\times2)$-matriser
    med koeffisienter i $\mathbb R$
    danner en gruppe under matrise multiplikasjon.
\end{example}

\todo{Bilde og kjerne}

\subsection{Oppgaver}
\begin{enumerate}
    \item Finn enhetselementet og inverselementene til alle eksemplene på
        grupper nevnt ovenfor.
    \item Vis at det bare finnes ett element $e\in G$ slik
        at $e \ast g = g\ast e = g$ for alle $g\in G$,
        altså at identitetselementet er unikt.
    \item Vis at for enhver $g\in G$ så finnes det bare ett element
        $h\in G$ slik at $gh = e$.
    \item Vis at for en morfi $f\colon G\to H$
        så har vi for alle $g\in G$ at $f(g^{-1}) = {f(g)}^{-1}$.
\end{enumerate}

\subsection{Mer om grupper}

\begin{definition}
    En gruppe $(G, \ast)$ er \textit{abelsk} om
    for alle $g,h\in G$ har vi $g\ast h = h\ast g$.
\end{definition}

\begin{example}
    Alle eksemplene er med addisjon som gruppeoperator ovenfor er abelske.
    Generelt, om vi benevner en operator med symbolet ``$+$'' så er operatoren
    kommutativ.
\end{example}

\begin{example}
    Gruppen av inverterbare $(2\times 2)$-matriser under multiplikasjon
    $\mathrm{GL}_2(\mathbb R)$
    er ikke abelsk, siden til eksempel
    \[
        \begin{bmatrix}
            1 & 1\\
            0 & 1
        \end{bmatrix}
        \begin{bmatrix}
            1 & 0\\
            1 & 1
        \end{bmatrix}
        =
        \begin{bmatrix}
            1 & 1\\
            1 & 2
        \end{bmatrix}
    \]
    mens
    \[
        \begin{bmatrix}
            1 & 0\\
            1 & 1
        \end{bmatrix}
        \begin{bmatrix}
            1 & 1\\
            0 & 1
        \end{bmatrix}
        =
        \begin{bmatrix}
            1 & 1\\
            1 & 2
        \end{bmatrix}.
    \]
\end{example}

\begin{definition}
    La $H\subset G$ være en undergruppe.
    Et \textit{coset} av $H$ i $G$ er en mengde på formen
    \[
        gH = \{gh\mid h\in H\}
    \]
    for en $g\in H$.
    Merk at $g\in gH$ siden $e\in H$.
\end{definition}

\begin{lemma}
    La $g, g^\prime$ være to elementer i $G$,
    og la $H\subset G$ være en undergruppe av $G$.
    Vi har
    \[
        gH = g^\prime H
    \]
    hvis og bare hvis $g^{-1} g^\prime\in H$.
\end{lemma}
\begin{proof}
    \todo[inline]{Bevis dette}
\end{proof}

\begin{corollary}
    Om $gH\cap g^\prime H\neq \emptyset$ så har vi $gH = g^\prime H$.
\end{corollary}

\begin{corollary}
    Cosetene til $H\subset G$ danner en partisjon av $G$.
\end{corollary}

\begin{definition}
    En delgruppe $H\subset G$ er \textit{normal}
    dersom
    \[
        gH = Hg = \{hg\mid h\in H\}
    \]
    for alle $g\in G$.
\end{definition}

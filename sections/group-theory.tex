\section{Grunnleggende gruppeteori}

\begin{definition}
    En \textit{gruppe} $(G,\ast)$ er en mengde $G$ sammen med en avbildning
    $\ast\colon G\times G\to G$.
    For to elementer $g, h\in G$ skriver vi $\ast(g,h)$ som $g\ast h$.
    Paret $(G, \ast)$ må tilfredsstille at
    \begin{itemize}
        \item det finnes et element $e\in G$ hvor for alle $g\in G$
            har vi $e\ast g = g\ast e = g$.
            Dette elementet er unikt med denne egenskapen (vis dette!)
            og kalles \textit{identitetselementet} $e_G = e \in G$.
        \item for $g, h, i\in G$ har vi
            $g\ast (h\ast i) = (g\ast h)\ast i$,
            altså $\ast$ er \textit{assosiativ}, og
        \item for hver $g$ finnes et element $h\in G$
            slik at $g \ast h = e$.
            Dette elementet $h$ er unikt med denne egenskapen (vis dette!)
            og vi benevner det $h = g^{-1}$, \textit{inverselementet}.
    \end{itemize}
    Vi sier at $\ast$ er \textit{gruppeoperatoren} til gruppen $(G,\ast)$.
\end{definition}

For å gjøre notasjonen enklere skriver vi ofte bare $G$ for gruppen $(G,\ast)$
når gruppeoperatoren er åpenbar.
Vi forkorter ofte også $g\ast h$ som $gh$.

\begin{definition}
    En avbildning $f\colon G\to H$ mellom to grupper
    er en \textit{morfi} om
    \begin{itemize}
        \item $f(e_G) = e_H$, og
        \item for alle $g, g^\prime\in G$ har vi $f(gg^\prime) = f(g)f(g^\prime)$.
    \end{itemize}
\end{definition}

\begin{definition}
    En delmengde $H\subset G$ av en gruppe $(G, \ast)$ er en \textit{undergruppe}
    om
    \begin{itemize}
        \item $e\in H$,
        \item for alle $h, h^\prime \in H$ er $h\ast h^\prime\in H$,
            og
        \item for alle $h\in H$ er $h^{-1}\in H$.
    \end{itemize}
\end{definition}

\begin{remark}
    Inklusjonen av en undergruppe $H\subset G$
    \[
        H\hookrightarrow G
    \]
    er en morfi.
\end{remark}

\begin{example}
    Vi kjenner allerede til mange eksempler på grupper,
    slik som kjeden av delmengder
    \[
        \mathbb Z
        \subset \mathbb Q
        \subset \mathbb R
        \subset \mathbb C
    \]
    hvor vi lar $\mathbb C = (\mathbb C, +)$ være en gruppe under addisjon
    danner en kjede av undergrupper.
    Merk at $\mathbb N$ ikke er en gruppe under addisjon siden det ikke finnes
    invers elementer.

    Om vi fjerner $0$ fra alle mengdene
    \[
        \mathbb Q^\times
        \subset \mathbb R^\times
        \subset \mathbb C^\times
    \]
    og utruster $\mathbb C$ med multiplikasjon isteden,
    så danner dette en kjede med undergrupper.
\end{example}

\begin{example}
    Om $(G, \ast_G), (H, \ast_H)$ er to grupper så danner $(G\times H, \ast)$
    en gruppe hvor $\ast$ er operatoren
    \[
        (g, h)\ast (g^\prime, h^\prime)
        = (g \ast_G g^\prime, h\ast_H h^\prime).
    \]
    Til eksempel er det reelle planet $\mathbb R^2$ en gruppe under vektor-addisjon.
\end{example}

\begin{example}
    La $\mathrm{Bij}(A)$ være mengden av bijeksjoner $A\to A$.
    Da er $(\mathrm{Bij}(A), \circ)$ en gruppe.
\end{example}

\begin{example}
    Mengden $\mathrm{Mat}_{2\times 2}$ av $(2\times 2)$-matriser under addisjon
    danner en gruppe.
\end{example}

\begin{example}
    Mengden $\mathrm{GL}_2(\mathbb R)$ av inverterbare $(2\times2)$-matriser
    med koeffisienter i $\mathbb R$
    danner en gruppe under matrise multiplikasjon.
\end{example}

\begin{example}
    La $f\colon G\to H$ være en morfi.
    Husk at for en avbildning definerte vi bildet av avbildningen
    $\im f\subset H$ som en delmengde.
    Denne delmengden er også en undergruppe.
\end{example}

La $f\colon G\to H$ være en morfi.
I tillegg til bildet $\im f\subset H$ kan vi lage en annen undergruppe.
\begin{definition}
    Vi definerer \textit{kjernen} til $f\colon G\to H$
    som
    \[
        \ker f = \{ g\in G\mid f(g) = e_H\} \subset G.
    \]
\end{definition}

\begin{lemma}
    Kjernen $\ker (f\colon G\to H)\subset G$ av en morfi er en undergruppe av $G$.
\end{lemma}
\begin{proof}
    Vi trenger å vise at $e_G\in \ker f$, at $\ker f$ er lukket under
    gruppe-operatoren, og at det finnes inverser.
    Den førstnevnte følger fra at $f(e_G) = e_H$ siden $f$ er en morfi.

    La $g,g^\prime \in\ker f$, da har vi $f(g) = f(g^\prime) = e_H$,
    men $f(gg^\prime) = f(g)f(g^\prime) = e_H$, så $gg^\prime\in \ker f)$.

    La $g\in \ker f$, så $f(g) = e_H$, men $f(g^{-1}) = {f(g)}^{-1} = e_H$,
    så $g^{-1}\in\ker f$.
\end{proof}

\begin{definition}
    La $(G, \ast_G), (H, \ast_H)$ være to grupper.
    Vi kan definere en operator $\ast_{G\times H}$
    på $G\times H$ ved
    \[
        (g, h)\ast_{G\times H} (g^\prime, h^\prime)
        = (g \ast_G g^\prime, h \ast_H h^\prime).
    \]
    Dette gjør $G\times H$ til en gruppe med identitetselement $(e_G, e_H)$.
\end{definition}

Som med mengder har vi fortsatt de to projeksjonene
\[\begin{tikzcd}
    G
    \rar[<-]{p}
    &
    G\times H
    \rar{q}
    &
    H,
\end{tikzcd}\]
men vi har også inklusjoner den andre veien
\[\begin{tikzcd}
    G
    \rar{\id_G\times e_H}
    &
    G\times H
    \rar[<-]{e_G\times \id_H}
    &
    H,
\end{tikzcd}\]
gitt ved $(\id_G\times e_H)\colon g\mapsto (g, e_H)$
og $(e_G\times \id_H)\colon h\mapsto (e_G, h)$.
Disse tilfredsstiller $p\circ (\id_G\times e_H) = \id_G$
og $q\circ (e_G\times \id_H) = \id_H$.
Merk også at $\im (\id_G\times e_H) = \ker q$.

\subsubsection*{Oppgaver}
\begin{enumerate}
    \item Finn enhetselementet og inverselementene til alle eksemplene på
        grupper nevnt ovenfor.
    \item Vis at det bare finnes ett element $e\in G$ slik
        at $e \ast g = g\ast e = g$ for alle $g\in G$,
        altså at identitetselementet er unikt.
    \item Vis at for enhver $g\in G$ så finnes det bare ett element
        $h\in G$ slik at $gh = e$.
    \item Vis at for en morfi $f\colon G\to H$
        så har vi for alle $g\in G$ at $f(g^{-1}) = {f(g)}^{-1}$.
    \item Vis at for en morfi $f\colon G\to H$ så er $\im f\subset H$ en undergruppe.
\end{enumerate}
Så langt har vi sett mange eksempler på uendelige grupper,
slik som $\mathbb C$ og dens undergrupper.
Vi ønsker å se på flere eksempler av \textit{endelige} grupper,
det vil si grupper hvor den underliggende mengden er endelig.
\begin{enumerate}[resume]
    \item La $S^1\subset \mathbb C$ være enhetsirkelen
        \[
            S^1 = \{ z\in \mathbb C\mid |z| = 1\}.
        \]
        \begin{enumerate}
            \item Vis at $S^1$ er en gruppe under (kompleks) multiplikasjon.
            \item La $n$ være et heltall og la $\mu_n\subset S^1$ være mengden av $n$-te
                enhetrøtter
                \[
                    \mu_n = \{
                        e^{\frac k n 2\pi i}
                        \mid k\in \mathbb Z
                    \}.
                \]
                Vis at $\# \mu_n = n$ og at $\mu_n$ er en delgruppe av $S^1$.
            \item Ta for deg $\mu_3, \mu_4$ og $\mu_{12}$.
                Vis at $f\colon \mu_{12}\to \mu_3\times \mu_4$
                gitt ved
                \[
                    e^{\frac k {12} 2\pi i}
                    \mapsto (e^{\frac k 3 2\pi i}, e^{\frac k 4 2\pi i})
                \]
                er en isomorfi.

                Det finnes en slik isomorfi $\mu_{nm}\to \mu_n\times \mu_m$
                så lenge $\mathrm{gcd}(n,m) = 1$.
            \item Vis at det ikke finnes noen isomorfi $\mu_4\to \mu_2\times\mu_2$.
        \end{enumerate}
    \item Ta for deg gruppen $\mathrm{Bij}(S)$ av bijeksjoner $S\to S$
        hvor $S$ er en endelig mengde.
        Vi så at om vi har en bijeksjon $f\colon S\to S^\prime$
        så er avbildningen
        $\tilde f\colon \mathrm{Bij}(S)\to \mathrm{Bij}(S^\prime)$
        gitt ved $\sigma\mapsto f\circ \sigma \circ f^{-1}$
        for hver $(\sigma\colon S\to S)\in\mathrm{Bij}(S)$
        en bijeksjon.
        \begin{enumerate}
            \item Vis at $\tilde f$ er en morfi.
                Det følger dermed at at $\tilde f$ er en isomorfi.
            \item Vis (ved induksjon) at om $S, S^\prime$ er to endelige mengder
                slik at $\# S = \# S^\prime$.
                Da finnes det en bijeksjon $S\to S^\prime$.
        \end{enumerate}
        Vi har nå sett at om to endelige mengder $S, S^\prime$
        har samme antall elementer så er gruppene av bijeksjoner
        $\mathrm{Bij}(S), \mathrm{Bij}(S^\prime)$ isomorfe.
        La $n = \# S$.
        Vi gir denne gruppen det generelle navnet
        \textit{gruppen av permutasjoner av $n$ elementer}
        $S_n = \mathrm{Bij}(\{1,\dots,n\})$.
        \begin{enumerate}[resume]
            \item Vis at $\# S_n = n!$.
        \end{enumerate}
    \item La $G$ være en endelig gruppe.
        En nært beslektet mengde til $S_n$ er mengden
        $\mathrm{Aut}(G) = \mathrm{Iso}(G, G)$
        av isomorfier $G\to G$, såkalte \textit{automorfier} av $G$.
        \begin{enumerate}
            \item Vis at $(\mathrm{Aut}(G), \circ)$ er en gruppe.
            \item Vis at om en bijeksjon $f\colon G\to H$ er en isomorfi,
                så er $\tilde f\colon \mathrm{Aut}(G)\to \mathrm{Aut}(H)$
                en isomorfi.
            \item Vis at $\mathrm{Aut}(\mu_4)$ ikke er isomorf
                med $\mathrm{Aut}(\mu_2\times\mu_2)$.
                [Merk at $\#\mathrm{Aut}(\mu_4) = 2$.]
        \end{enumerate}
\end{enumerate}


\section{Anvendelser}

\subsection{Diffie-Hellman}
\subsection{RSA}

RSA (Rivest-Shamir-Adleman) \cite{rsa} er en metode for asymmetrisk kryptering
basert på modulær aritmetikk.
Algoritmen for kryptering på Alice sin side er som følger:
\begin{enumerate}
    \item Velg to primtall $p, q$ og la $n = pq$.
    \item Velg en $e\in 3,\dots, (p - 1)(q - 1) - 3$ slik at
        $\mathrm{gcd}(e, (p - 1)(q - 1)) = 1$.
    \item Beregn $d$ slik at $e d \equiv 1\mod (p - 1)(q - 1)$.
    \item La $m$ være meldingen til Alice gitt ved et tall $m \in \mathbb Z / n$.
        Beregn chifferteksten som $M = m^e \mod n$.
    \item Alice publiserer chifferteksten $M$ sammen med tallet $n$.
        Tallet $d$ sendes hemmelig og kan brukes til å dekryptere $M$.
\end{enumerate}
Gitt $n, M$ og $d$ kan vi enkelt dekryptere meldingen ved å beregne
$m^\prime = M^d \mod n$.
Siden $M^d = {(m^e)}^d = m^{ed} = m$ ved \cref{thm:fermat}
og $\cref{thm:chinese-remainder}.

\susubsection{Eulers totient-funksjon}
Men hvorfor regner vi modulo $(p - 1)(q - 1)$?
Tallet kommer av at \textit{Eulers totient-funksjon} $\phi$.
For et heltall $n$ har vi
\[
    \phi(n) = \#\{m\mid 0\leq m < n,\, \mathrm{gcd}(n, m) = 1\}.
\]
Merk at dette er det samme som å si at restklassen til $m$
i $\mathbb Z / n$ har en multiplikativ invers ved Euclids algoritme,
så $\phi(n) = {(\mathbb Z / n)}^\times$.

For et primtall $p$ har vi åpenbart at $\phi(p) = p - 1$
siden alle tallene mindre en tallet selv er relativt primisk til $p$.
For et produkt av primtall $n = pq$ så er et tall $m$ ikke relativt primisk
hvis og bare hvis $p | m$ eller $q | m$,
det vil si
\[
    m\in\{p, 2p, \dots, (q - 1)p\}\cup \{q, 2q, \dots, (p - 1)q\},
\]
så da gjenstår $(pq - 1) - (p - 1) - (q - 1) = (p - 1)(q - 1)$ tall,
så $\phi(pq) = (p - 1)(q - 1)$.

Merk at dette henger sammen med at ${(\mathbb Z / p)}^\times \cong \mathbb Z / (p - 1)$
og ${(\mathbb Z / pq)}^\times\cong \mathbb Z / (p - 1)\times \mathbb Z / (q - 1)$.

Merk at vi kan jobbe med en enda mindre gruppe,
for for et valg av melding $m$ jobber vi bare i den syklsiske undergruppen
$\langle m\rangle \subset {(\mathbb Z / n)}^\times$.
I beste fall er dette den største syklsike undergruppen.
For $n = pq$ har denne orden $\lambda(n) = \mathrm{lcm}(p - 1, q - 1)$ ved
\cref{thm:chinese-remainder}, og størrelsen på alle andre sykler deller
$\lambda(n)$, så det rekker å regne modulo $\lambda(n) \leq\phi(n)$.
Funksjonen $\lambda$ som gir oss størrelsen på den største syklen
i ${(\mathbb Z / n)}^\times$ for en vilkårlig $n$ kalles
\textit{Carmichaels totient-funksjon}.

\subsection{Diskrete logaritmer}
\subsubsection{Kinesisk restteorem}
\todo[inline]{
    Beskriv diskret logaritme-problemet på additive grupper isteden.
    Å løse DLP på additive grupper er trivielt, men å løse dem på
    til eksempel $\mathbb Z / p$ er vanskelig, fordi vi ikke
    har en eksplisitt beskrivelse av isomorfien.
    Om vi har en eksplisitt isomorfi er problemet enkelt,
    og vi kan få en eksplisitt beskrivelse av isomorfien
    ved å løse DLP for nok elementer,
    sa på denne måten er de to problemene ekvivalente.
}
Diskret logaritme-problemet for en syklisk gruppe $\mathbb Z / n$
handler om at man har to elementer $a, b\in (§\mathbb Z / n)^\times$
og man vil regne ut en eksponent $r$ slik at $a^r = b$.
Merk at dette ikke alltid er mulig.
\begin{example}
    La $p = 5$, og la $a = 4$ og $b = 2$ i ${(\mathbb Z / 5)}^\times$.
    Da har vi $\langle a\rangle = \{1, 4\}$,
    så det finnes ingen potens av $a$ som gir $b$.
\end{example}

Diskret logaritme-problemet er viktig for sikkerheten av både
Diffie-Hellman og RSA.
Flere andre kryptografiske metoder baserer seg også på de samme prinsippene
som Diffie-Hellman og RSA.

Merk at noen valg av $n$ gjør diskret logaritme-problemet lett å løse.
Anta det finnes en $r$ slik at $a^r = b$ for $a,b\in {(\mathbb Z / n)}^\times$.
La $d | (n - 1)$.
Da vet vi ved \cref{thm:fermat} at $\left(a^{\frac{p - 1} d}\right)^d = 1$,
så ved å beregne $b^{\frac {p - 1} d}$ lar det oss regne ut
$\bar r\in \mathbb Z / d$ om vi kan løse diskret logaritme-problemet
på
\subsubsection{Pollards rho}
\subsection{Miller-Rabin}

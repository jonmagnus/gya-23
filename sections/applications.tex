\section{Anvendelser}

\subsection{Diffie-Hellman}

\subsection{Diskrete logaritmer}\label{sec:discrete-logarithms}
\subsubsection{Kinesisk restteorem}
\todo[inline]{
    Beskriv diskret logaritme-problemet på additive grupper isteden.
    Å løse DLP på additive grupper er trivielt, men å løse dem på
    til eksempel $\mathbb Z / p$ er vanskelig, fordi vi ikke
    har en eksplisitt beskrivelse av isomorfien.
    Om vi har en eksplisitt isomorfi er problemet enkelt,
    og vi kan få en eksplisitt beskrivelse av isomorfien
    ved å løse DLP for nok elementer,
    sa på denne måten er de to problemene ekvivalente.
}
Diskret logaritme-problemet for en syklisk gruppe $\mathbb Z / n$
handler om at man har to elementer $a, b\in (\mathbb Z / n)^\times$
og man vil regne ut en eksponent $r$ slik at $a^r = b$.
Merk at dette ikke alltid er mulig.
\begin{example}
    La $p = 5$, og la $a = 4$ og $b = 2$ i ${(\mathbb Z / 5)}^\times$.
    Da har vi $\langle a\rangle = \{1, 4\}$,
    så det finnes ingen potens av $a$ som gir $b$.
\end{example}

Diskret logaritme-problemet er viktig for sikkerheten av både
Diffie-Hellman og RSA.
Flere andre kryptografiske metoder baserer seg også på de samme prinsippene
som Diffie-Hellman og RSA.

Merk at noen valg av $n$ gjør diskret logaritme-problemet lett å løse.
Anta det finnes en $r$ slik at $a^r = b$ for $a,b\in {(\mathbb Z / n)}^\times$.
La $d | (n - 1)$.
Da vet vi ved \cref{thm:fermat} at $\left(a^{\frac{p - 1} d}\right)^d = 1$,
så ved å beregne $b^{\frac {p - 1} d}$ lar det oss regne ut
$\bar r\in \mathbb Z / d$ om vi kan løse diskret logaritme-problemet
på
\subsubsection{Pollards rho}
\subsection{Miller-Rabin}

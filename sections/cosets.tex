
\section{Restklasser}

\begin{definition}
    La $H\subset G$ være en undergruppe.
    En \textit{restklasse} av $H$ i $G$ er en mengde på formen
    \[
        gH = \{gh\mid h\in H\}
    \]
    for en $g\in H$.
    Merk at $g\in gH$ siden $e\in H$.
\end{definition}

\begin{example}
    La $G = \mathbb Z$ være gruppen av heltall under addisjon.
    Ta for deg undergruppen $H = 2\mathbb Z = \{ 2n\mid n\in \mathbb Z\}$
    av partall.
    Partallene har to restklasser i $\mathbb Z$,
    nemlig mengden vi får når vi legger til et partall $2m$
    \[
        2m + (2H) = \{2m + 2n = 2(m + n)\mid n\in \mathbb Z\} = 2\mathbb Z
    \]
    som blir partallene selv,
    og mengden vi får når vi legger til et oddetall
    \[
        (2m + 1) + (2H) = \{2m + 1 + 2n
        = 2(m + n) + 1\mid n\in \mathbb Z\} = 1 + 2\mathbb Z
    \]
    som blir alle oddetallene.
\end{example}

\begin{lemma}
    La $g, g^\prime$ være to elementer i $G$,
    og la $H\subset G$ være en undergruppe av $G$.
    Vi har
    \[
        gH = g^\prime H
    \]
    hvis og bare hvis $g^{-1} g^\prime\in H$.
\end{lemma}
\begin{proof}
    Anta at $g H = g^\prime H$,
    så for alle $h\in H$ finnes en $h^\prime\in H$
    slik at $gh = g^\prime h^\prime$,
    men da kan vi regne
    \[\begin{aligned}
        gh &= g^\prime h^\prime \\
        g^{-1} gh = h
           &= g^{-1} g^\prime h^\prime \\
        h {(h^\prime)}^{-1}
           &= g^{-1} g^\prime h^\prime {(h^\prime)}^{-1}
            = g^{-1} g^\prime
    \end{aligned}\]
    så $g^{-1}g^\prime = h {(h^\prime)}^{-1}\in H$ siden $H$ er en undergruppe.

    For den andre veien anta at $g^{-1} g^\prime \in H$.
    Anta for motsigelse at $gH\neq g^\prime H$,
    så ved symmetri kan vi anta (``uten tap av generalitet'')
    at det finnes et element $gh\in gH$
    slik at $gh\notin g^\prime H$.
    Men $g^{-1} g^\prime \in H$,
    så $h^\prime = {(g^{-1} g^\prime)}^{-1} h\in H$,
    så $gh = g (g^{-1}g^\prime) h^\prime = g^\prime h^\prime \in g^\prime H$,
    men vi antok at $gh\notin g^\prime H$, så vi har en motsigelse.
\end{proof}

\begin{corollary}\label{thm:coset-no-partial-intersection}
    Om $gH\cap g^\prime H\neq \emptyset$ så har vi $gH = g^\prime H$.
\end{corollary}

\begin{corollary}
    Restklassene til $H\subset G$ danner en \textit{partisjon} av $G$,
    det vil si vi kan finne en \textbf{delmengde} (ikke en gruppe) elementer
    $S\subset G$ slik at $gH \cap g^\prime H = \emptyset$ for alle $g,g^\prime\in S$
    med $g\neq g^\prime$, og
    \[
        \bigcup_{g\in S} gH = G.
    \]
\end{corollary}
\begin{proof}
    La $\mathscr S$ være mengden delmengder $S\subset G$ slik at
    for alle $g,g^\prime \in G$ så har vi $gH\cap g^\prime = \emptyset$
    for $g\neq g^\prime$.
    Anta $\tilde S\in \mathscr S$ er maksimal, det vil si det finnes ingen
    $S\in \mathscr S$ slik at $\tilde S\subsetneq S$.
    Vi påstår at $\bigcup_{g\in \tilde S} gH = G$.
    Anta for motsigelse at det finnes en $g^\prime\in G$ slik at
    $g^\prime\notin \bigcup_{g\in \tilde S} gH$.
    Da har vi at $g^\prime H\cap gH = \emptyset$ for alle $g\in \tilde S$,
    for ellers har vi $g^\prime \in gH$ ved \cref{thm:coset-no-partial-intersection},
    men da har vi at $\tilde S\cup \{g^\prime\}\in\mathscr S$
    som motsier at $\tilde S$ er maksimal,
    så det finnes ingen slik $g^\prime$ og $\bigcup_{s\in\tilde S} gH = G$.
\end{proof}

\begin{remark}
    Hvordan vet vi egentlig at det i det hele tatt finnes en maksimal
    mengde $\tilde S$ i $\mathscr S$?
    Om gruppen er uendelig kan man tenke seg at vi alltid
    har plass til å legge til flere og flere elementer til $\tilde S$.
    Det som redder oss er at vi vet at $\tilde S$ er ihvertfall mindre enn $G$,
    så vi kan på en litt innviklet måte bruke et ganske abstrakt aksiom
    kalt \textit{Zorns lemma} til å vite at det finnes et maksimalt element,
    men vi vet ikke nødvendigvis hvordan vi skal finne et slikt element!
\end{remark}

\subsection{Kvotientgruppen}

\begin{lemma}
    La $G$ være en gruppe, $H\subset G$ en undergruppe
    og $g\in G$ et element.
    Da er $\# gH = \# H$.
\end{lemma}
\begin{proof}
    Vi har en naturlig avbildning $f\colon H\to gH$ gitt ved $h\mapsto gh$.
    Denne er surjektiv per definisjon av $gH$, så det gjenstår å vise at den er
    injektiv.

    Anta at $f(h) = f(h^\prime)$ for to $h, h^\prime \in H$,
    altså at $gh = gh^\prime$, men da har vi at
    \[
        h
        = g^{-1} f(h)
        = g^{-1} f(h^\prime)
        = h^\prime.
    \]
\end{proof}

\begin{definition}
    La $G$ være en gruppe og $H\subset G$ en undergruppe.
    Vi definerer mengden av restklasser
    \[
        G / H = \{ gH \mid g\in G\}.
    \]
\end{definition}

\begin{remark}
    Merk at avbildningen $G\to G / H$ gitt ved $g\mapsto gH$ ikke er injektiv,
    for det er flere ulike $g\neq g^\prime$ slik at $gH = g^\prime H$,
    og dette er akkurat de parene $(g, g^\prime)$ slik at $g^{-1}g^\prime\in H$.
\end{remark}

\begin{example}
    La $H\subset G$ være endelige grupper.
    Vi har allerede sett at restklassene danner en partisjon av $G$,
    og nå har vi sett at alle restklassene er like store.
    Det følger umiddelbart at vi må ha
    \[
        \# G = \# (G / H) \# H,
    \]
    altså
    \[
        \# G / \# H = \# (G / H)
    \]
    som motiverer notasjonen.
\end{example}

Vi kan tenke oss en naturlig gruppeoperator på mengden $G / H$,
nemlig at for to restklasser $gH$ og $g^\prime H$ definerer vi produktet deres som
\[
    (gH)(g^\prime H) = (gg^\prime)H,
\]
men husk at vi har flere elementer $\hat g\in G$ som gir samme restklasse
$\bar gH = gH$ enda $\bar g\neq g$.
Så for at denne operatoren skal være ``veldefinert'' på $G / H$ trenger
vi at $(\bar gg^\prime) H = (g g^\prime)H$,
det vil si at
${(g g^\prime)}^{-1} (\bar g g^\prime) = {(g^\prime)}^{-1} h g^\prime\in H$
hvor $h = g^{-1} g\in H$,
men dette er ikke automatisk!

\begin{example}
    Ta for deg gruppen av permutasjoner $G = S_3$ av mengden på tre elementer $\{1,2,3\}$,
    og la $H$ være undergruppen bestående av identiteten og permutasjonen
    \[
        (12)\colon \begin{cases}
            1\mapsto 2\\
            2\mapsto 1\\
            3\mapsto 3.
        \end{cases}
    \]
    La $g= (13)$ og $g^\prime (23)$.
    Vi ser at $gH = \{(13)e, (13)(12)\} = \{(13), (123)\}$,
    så $\bar gH = gH$ hvor $\bar g = (123)$,
    men $gg^\prime = (13)(23) = (321)$,
    mens $\bar g g^\prime = (123)(23) = (12)$,
    så
    \[
        (gg^\prime)H = (321)H\neq H = (12)H = (\bar g g^\prime)H.
    \]
\end{example}

\begin{definition}
    En undergruppe $H\subset G$ er \textit{normal}
    dersom
    \[
        gH = Hg = \{hg\mid h\in H\}
    \]
    for alle $g\in G$.
\end{definition}

\begin{lemma}
    Om $H\subset G$ er en normal undergruppe så er ``multiplikasjon''
    veldefinert på mengden av restklasser $G / H$,
    så $G / H$ danner en gruppe -- \textit{kvotientgruppen} av $G$ over $H$.
\end{lemma}
\begin{proof}
    Om vi har $gH = Hg$ for alle $g\in H$ så følger det at
    $H = g^{-1} H g$,
    så $g^{-1}h g\in H$ for alle $h\in H$.
\end{proof}

\begin{lemma}
    La $f\colon G\to H$ være en morfi.
    Da er $\ker f\subset G$ en normal undergruppe.
\end{lemma}
\begin{proof}
    La $vg\in(\ker f) g$.
    Vi ønsker å vise at $vg\in g\ker f$,
    men det er det samme som at $g^{-1} vg\in \ker f$.
    Vi ser at
    \[\begin{aligned}
        f(g^{-1}vg)
        &= f(g^{-1})f(v)
        \\
        &= {(f(g))}^{-1} e_H f(g)
        \\
        &= e_H,
    \end{aligned}\]
    så $g^{-1} vg\in \ker f$.
\end{proof}

\begin{theorem}[Isomorfiteoremet]\label{thm:isomorphism-theorem}
    La $f\colon G\to H$ være en surjektiv morfi.
    Da har vi en isomorfi $\hat f\colon G / \ker f \to H$,
    det vil si vi kan fylle inn følgende avbildning
    \[\begin{tikzcd}
        G
        \rar
        \drar[swap]{f}
        &
        G / \ker f
        \dar[dashed]{\bar f}
        \\
        &
        H.
    \end{tikzcd}\]
\end{theorem}
\begin{proof}
    Vi benevner restklassen $g + \ker f$ i $G / \ker f$
    ved $\bar g$.
    Vi konstruerer en avbildning $\bar f\colon G / \ker f\to H$
    ved at for en $\bar g\in G / \ker f$ setter vi
    $\bar f(\bar g) = f(g)$ for en representant $g$.

    Om vi velger en annen representant $g^\prime$ med
    $\bar g^\prime = \bar g$ har vi at $g^{-1}g^\prime\in\ker f$,
    så $f(g) = f(g)f(g^{-1}g^\prime) = f(gg^{-1})f(g^\prime) = f(g^\prime)$,
    så definisjonen av $\bar f$ er uavhengig av hvordan vi velger representanter.

    For alle $h\in H$ så finnes en $g\in G$ slik at $f(g) = h$,
    så $\bar f(\bar g) = h$, og $\bar f$ er surjektiv.

    La $g, g^\prime$ slik at $f(g) = f(g^\prime)$.
    Da er $f(g^{-1}g^\prime) = f(g)^{-1} f(g^\prime) = e$,
    så $g^{-1}g^\prime\in \ker f$, så $\bar g = \bar g^\prime$.

    Det gjenstår å vise at $\bar f$ er en morfi,
    det vil si at for alle $\bar g, \bar g^\prime \in G / \ker f$
    så er $\bar f(\bar g\bar g^\prime) = \bar f(\bar g)\bar f(\bar g^\prime)$.
    Vi har at $\bar g\bar g^\prime = \overline{(gg^\prime)}$,
    og vi vet allerede at dette er veldefinert siden $\ker f$ er normal,
    så det gjenstår bare å sjekke at $f(g)f(g^\prime) = f(gg^\prime)$
    for et vilkårlig valg av representanter $g, g^\prime$,
    men dette følger av at $f$ er en morfi.
\end{proof}

\subsubsection*{Oppgaver}

\begin{enumerate}
    \item
        \begin{enumerate}
            \item Hvilke restklasser har $3\mathbb Z = \{3n \mid n\in \mathbb Z\}$
                i $\mathbb Z$? Hva med $5\mathbb Z\subset \mathbb Z$?
            \item For et generelt heltall $n\in \mathbb Z$,
                vis at $n\mathbb Z\subset \mathbb Z$ har $n$ restklasser,
                det vil si $\#(\mathbb Z / n\mathbb Z = n$.
        \end{enumerate}
    \item Bevis \cref{thm:coset-no-partial-intersection}
    \item Vi skal undersøke mengden av restklasser til $\mathbb Z\subset \mathbb R$.
        \begin{enumerate}
            \item Vi har sett at to restklasser $a + \mathbb Z = b + \mathbb Z$
                hvis og bare hvis $(a - b)\in \mathbb Z$.
                Bruk dette til å konstruere en bijeksjon
                $[0,1)\to \mathbb R / \mathbb Z$.
            \item Vi har at $a + \mathbb Z = \mathbb Z + a$,
                så $\mathbb Z\subset \mathbb R$ er en normal undergruppe
                og $\mathbb R / \mathbb Z$ danner en gruppe.
                Vis at avbildningen
                \[
                    \mathrm{exp}\colon \begin{cases}
                        (\mathbb R, +)\mapsto (S^1,\times)\\
                        x\mapsto e^{2\pi i x}
                    \end{cases}
                \]
                er en gruppe-morfi (merk at vi går fra addisjon til multiplikasjon)
                og at den danner en bijeksjon $[0,1)\to S^1$.
                Her er $S^1$ enhetsirkelen
                \[
                    S^1 = \{z\mid |z| = 1\}\subset \mathbb C.
                \]
                Dette viser at vi har en isomorfi $\mathbb R / \mathbb Z \to S^1$.
            \item Kan du forestille deg hva som skjer om vi tar
                $\mathbb R / \mathbb Q$?
        \end{enumerate}
\end{enumerate}



\section{RSA}

RSA (Rivest-Shamir-Adleman) \cite{rivest_method_1978} er en metode for asymmetrisk kryptering
basert på modulær aritmetikk.
Algoritmen for kryptering på Alice sin side er som følger:
\begin{enumerate}
    \item Velg to primtall $p, q$ og la $n = pq$.
    \item Velg en $e\in 3,\dots, (p - 1)(q - 1) - 3$ slik at
        $\mathrm{gcd}(e, (p - 1)(q - 1)) = 1$.
    \item Beregn $d$ slik at $e d \equiv 1\mod (p - 1)(q - 1)$.
    \item La $m$ være meldingen til Alice gitt ved et tall $m \in \mathbb Z / n$.
        Beregn chifferteksten som $M = m^e \mod n$.
    \item Alice publiserer chifferteksten $M$ sammen med tallet $n$.
        Tallet $d$ sendes hemmelig og kan brukes til å dekryptere $M$.
\end{enumerate}
Gitt $n, M$ og $d$ kan vi enkelt dekryptere meldingen ved å beregne
$m^\prime = M^d \mod n$.
Siden $M^d = {(m^e)}^d = m^{ed} = m$ ved \cref{thm:fermat}
og \cref{thm:chinese-remainder}.

\subsection{Eulers totient-funksjon}
Men hvorfor regner vi modulo $(p - 1)(q - 1)$?
Tallet kommer av at \textit{Eulers totient-funksjon} $\phi$.
For et heltall $n$ har vi
\[
    \phi(n) = \#\{m\mid 0\leq m < n,\, \mathrm{gcd}(n, m) = 1\}.
\]
Merk at dette er det samme som å si at restklassen til $m$
i $\mathbb Z / n$ har en multiplikativ invers ved Euclids algoritme,
så $\phi(n) = {(\mathbb Z / n)}^\times$.

For et primtall $p$ har vi åpenbart at $\phi(p) = p - 1$
siden alle tallene mindre en tallet selv er relativt primisk til $p$.
For et produkt av primtall $n = pq$ så er et tall $m$ ikke relativt primisk
hvis og bare hvis $p | m$ eller $q | m$,
det vil si
\[
    m\in\{p, 2p, \dots, (q - 1)p\}\cup \{q, 2q, \dots, (p - 1)q\},
\]
så da gjenstår $(pq - 1) - (p - 1) - (q - 1) = (p - 1)(q - 1)$ tall,
så $\phi(pq) = (p - 1)(q - 1)$.

Merk at dette henger sammen med at ${(\mathbb Z / p)}^\times \cong \mathbb Z / (p - 1)$
og ${(\mathbb Z / pq)}^\times\cong \mathbb Z / (p - 1)\times \mathbb Z / (q - 1)$.
Det følger at vi kan jobbe med en enda mindre gruppe,
for for et valg av melding $m$ jobber vi bare i den syklsiske undergruppen
$\langle m\rangle \subset {(\mathbb Z / n)}^\times$.
I beste fall er dette den største syklsike undergruppen.
For $n = pq$ har denne orden $\lambda(n) = \mathrm{lcm}(p - 1, q - 1)$ ved
\cref{thm:chinese-remainder}, og størrelsen på alle andre sykler deller
$\lambda(n)$, så det rekker å regne modulo $\lambda(n) \leq\phi(n)$.
Funksjonen $\lambda$ som gir oss størrelsen på den største syklen
i ${(\mathbb Z / n)}^\times$ for en vilkårlig $n$ kalles
\textit{Carmichaels totient-funksjon}.

\subsection{Sikkerhet ved signering}

Det er flere tenkelige måter å angripe RSA på.
Om Alice bruker RSA som en metode for å ``signere'' så kan
vi som angriper anta at vi vet $n, m, M$ og $d$.
Målet vårt da er å finne $e$ slik at vi kan signere meldinger med Alice
sin hemmelige nøkkel.

Om vi vet $p$ og $q$ så kan vi enkelt regne regne ut $e$ ved hjelp av $d$
og Euclids algoritme
slik Alice gjorde for å finne $d$ fra $e$ i RSA.
Dette er ikke noe problem siden vi kjenner bare til produktet $n = pq$.
Alternativt kan vi forsøke å beregne $\phi(n)$ direkte fra $n$,
men $\phi(n)$ er nødvendigvis også vanskelig å beregne,
for om vi vet $\phi(n)$ kan vi enkelt beregne $p$ og $q$.

Merk at om vi lar $n = p$, altså velger $n$ som et primtall isteden for et
produkt av primtall er det trivielt å finne $\phi(n) = n - 1$,
så får sikkerheten av algoritmen er det viktig at $n$ er et sammensatt tall.
Om vi derimot lar $n$ være et produkt av flere primtall må vi sørge for
at primtallene er store nok slik at vi ikke kan enkelt finne faktorer.

\begin{example}
    La $n$ være et vilkårlig sammensatt tall, og anta vi kjenner til
    en faktor $p | n$.
    Da har vi at $\phi(p) = p - 1 | \phi(n)$.
    Faktisk vet vi at $p - 1 | \lambda(n)$.
    La $e, d$ og $M$ være slik som beregnet i RSA, og anta vi kjenner til $d$ og $M$.
    Vi har at $de \equiv 1\mod (p - 1)$.
    La $e^\prime\in {(\mathbb Z / p)}^\times$ slik at $de^\prime\equiv 1 \mod (p - 1)$,
    så ved \cref{thm:chinese-remainder} har vi at
    \[
        e = e^\prime(1 + k(p - 1))
    \]
    for en $k\in \{0,\dots, \lambda(n) / (p - 1)\}$
    siden
    \[\begin{aligned}
        m^{e^\prime(1 + k(p - 1)) d}
        &= {(m^{e^\prime d})}^{1 + k(p - 1)}
        \\
        &\equiv m^{1 + k(p - 1)}
        & \mod (p - 1)
        \\
        &= m {(m^{p - 1})}^k
        \\
        &\equiv m
        &\mod (p - 1).
    \end{aligned}\]
    Antall $k$ som må testes synker enda mer om $\#\langle m\rangle < \lambda(n)$.
\end{example}

Vi kan også finne $e$ ved å løse ``discret logaritmeproblemet'' $e = \log_{m}(M)$
i $\mathbb Z / n$,
se \cref{sec:discrete-logarithms}.

\subsection{Sikkerhet ved kryptering}

Alice kan også bruke RSA som en offentlig krypteringsalgoritme,
dvs.\ at Alice oppgir $e$ og holder $d$ hemmelig slik at alle andre
kan beregne en chiffertekst $M = m^e$ men bare Alice kan finne tilbake meldingen
$m = M^d$.
Isåfall kan vi anta at vi kjenner til $e$ og $n$ fra Alice,
og en hemmelig melding $M$ fra Bob.
Isåfall er det interessant å bare finne $m = M^d$ for å lytte på samtalen til
Alice og Bob.
Om vi klarer å finne $d$ så er resten enkelt,
men dette er enda vanskeligere enn å låse signeringsproblemet,
siden vi ikke vet renteksten $m$.
Vi søker derfor etter en metode som bare finner renteksten $m$ i håp om at dette
er enklere.

Altså trenger vi å finne $e$-te roten av $M$.
Om $m$ tilfredsstiller noen mildere betingelser finnes en
metode som kan beregne denne roten, kjent som Coppersmiths metode
\cite{coppersmith_finding_1996}.
Det metoden gjør spesifikt er at om vi har et polynom $p(x)$
slik at $p(x_0) = 0 \mod n$ for en $x_0$ tilstrekkelig liten
så kan vi finne et annet polynom $q(x)$ med så små koeffisienter
at $q(x_0) = 0$ uten å redusere modulo $n$.
Når vi har et slikt polynom kan vi finne $x_0$ med klassiske numeriske metoder,
som til eksempel Newtons metode.

\section{Endelige abelske grupper}

\subsection{Fermats lille teorem}

\begin{lemma}\label{thm:binomial-identity}
    La $x,y\in \mathbb Z$  og $n \geq 1$ være heltall.
    Da har vi formelen
    \[
        (x + y)^n = \sum_{i = 0}^n \binom n i x^i y^{n - i}.
    \]
\end{lemma}

Av og til tar vi dette som definisjonen av \textit{binomialkoeffisientene}
$\binom n m$,
men om vi heller bruker definisjonen fra Pascals trekant $\binom n 0 = \binom n n = 1$
for alle $n\in\mathbb N$ og $\binom n {m - 1} + \binom n m = \binom {n + 1} m$
for alle $n, m$ kan vi vise \cref{thm:binomial-identity} ved induksjon som følger.
\begin{proof}
    Tilfellet $n = 1$ er enkelt siden
    $(x + y)^1 = x + y$ og $\binom 1 0 = \binom 1 1 = 1$.

    Anta $(x + y)^k = \sum_{i = 0}^k \binom k i x^i y^{k - i}$.
    Vi har
    \[\begin{aligned}
        (x + y)^{k + 1}
        &=  (x + y)(x + y)^k
        \\
        &= (x + y)\sum_{i = 0}^k \binom k i x^i y^{k - i}
        \\
        &= \sum_{i = 0}^k \binom k i x^{i + 1} y^{k - i}
        + \sum_{i = 0}^k \binom k i x^{i} y^{k - i + 1}
        \\
        &= \sum_{i = 1}^{k + 1} \binom k {i - 1} x^{i} y^{(k + 1) - i}
        + \sum_{i = 0}^k \binom k i x^{i} y^{(k + 1) - i}
        \\
        &= \sum_{i = 0}^{k + 1} \binom {k + 1} i x^{i} y^{(k + 1) - i}
    \end{aligned}\]
\end{proof}

\begin{corollary}{Fermats lille teorem}
    La $p$ være et primtall.
    Da har vi at
    \[
        a^p \cong a \mod p
    \]
    for alle $a\in \mathbb Z$.
\end{corollary}
\begin{proof}
    Vi definerer en avbildning $\pi_p\colon \mathbb Z / p\to\mathbb Z / p$
    ved $a\mapsto a^p$.
    La $a,b\in \mathbb Z$.
    Vi ser at
    \[\begin{aligned}
        \pi_p(a + b)
        &= (a + b)^p
        \\
        &= \sum_{i = 0}^p \binom p i a^i b^{p - i}
        \\
        &= a^p + b^p
        \\
        &= \pi_p(a) + \pi_p(b)
    \end{aligned}\]
    siden $p|\binom p i$ for $i\neq 0, p$,
    så $\pi_p$ er en morfi.
    Vi ser også at $\pi_p(1) = 1^p = 1$,
    og det er bare \`en morfi $\mathbb Z / p\to \mathbb Z / p$
    som tilfredsstiller dette, nemlig identiteten $\id_{\mathbb Z / p}$,
    så $\pi_p = \id_{\mathbb Z / p}$.
\end{proof}

\subsection{Klassifisering av endelige abelske grupper}
Vi har allerede sett at alle endelige sykliske grupper kan skrives på formen
$\mathbb Z / n$ for et heltall $n$.
Vi ønsker å generalisere dette til alle endelige abelske grupper.
Vi har sett at det finnes endelige abelske grupper som ikke er sykliske
slik som $\mathbb Z / 2\times \mathbb Z / 2$ og genereres av minst to elementer.
Andre grupper som vi genererer av to elementer kan vise seg å være sykliske derimot,
slik som $\mathbb Z / 2\times \mathbb Z / 3 \cong \mathbb Z / 6$.
Som en konsekvens av \cref{thm:chinese-remainder} kan vi vise
\begin{corollary}
    La $p_1,\dots,p_n$ være primtall slik at $p_i \neq p_j$
    for alle $i\neq j$,
    og la $r_1,\dots,r_n$ være heltall.
    Da har vi en isomorfi
    \[
        \mathbb Z / (p_1^{r_1}\dots p_n^{r_n})
        \cong
        \mathbb Z / p_1^{r_1}\times\dots\times\mathbb Z / p_n^{r_n}.
    \]
\end{corollary}

Vi ønsker noe i denne stilen som tar hånd om alle endelige abelske grupper.
\begin{theorem}
    La $A$ være en endelig abelsk gruppe med $n = \# A$.
    Da finnes det primtall $p_1,\dots, p_m$,
    eksponenter $r_1,\dots,r_m$,
    og en isomorfi
    \[
        \mathbb Z / p_1^{r_1} \times \dots \times\mathbb Z / p_m^{r_m}
        \to A.
    \]
\end{theorem}
\begin{proof}
    \todo[inline]{Bevis dette ved å velge et minste antall generatorer
    og bruk induksjon}
\end{proof}

\subsubsection*{Oppgaver}

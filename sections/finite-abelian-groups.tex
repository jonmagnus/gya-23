\section{Endelige abelske grupper}

\begin{definition}
    En gruppe $(G, \ast)$ er \textit{abelsk} om
    for alle $g,h\in G$ har vi $g\ast h = h\ast g$.
    En operator som tilfredsstiller dette kalles \textit{kommutativ}.
\end{definition}

\begin{example}
    Alle eksemplene er med addisjon som gruppeoperator ovenfor er abelske.
    Generelt, om vi benevner en operator med symbolet ``$+$'' så er operatoren
    kommutativ.
\end{example}

\begin{example}
    Gruppen av invertible $(2\times 2)$-matriser under multiplikasjon
    $\mathrm{GL}_2(\mathbb R)$
    er ikke abelsk, siden til eksempel
    \[
        \begin{bmatrix}
            1 & 1\\
            0 & 1
        \end{bmatrix}
        \begin{bmatrix}
            1 & 0\\
            1 & 1
        \end{bmatrix}
        =
        \begin{bmatrix}
            1 & 1\\
            1 & 2
        \end{bmatrix}
    \]
    mens
    \[
        \begin{bmatrix}
            1 & 0\\
            1 & 1
        \end{bmatrix}
        \begin{bmatrix}
            1 & 1\\
            0 & 1
        \end{bmatrix}
        =
        \begin{bmatrix}
            2 & 1\\
            1 & 1
        \end{bmatrix}.
    \]
\end{example}

\begin{definition}
    La $g_1,\dots, g_n\subset G$ være et endelig antall elementer i $G$.
    Vi definerer \textit{gruppen generert av $g_1,\dots, g_n$}
    \[
        H = \langle g_1,\dots,g_n\rangle \subset G
    \]
    som den minste undergruppen av $G$ som inneholder alle elementene $g_1,\dots,g_n$.

    Om det finnes en $g\in G$ slik at $G = \langle g\rangle$ sier vi at $G$
    er en \textit{syklisk} gruppe.
\end{definition}

\begin{remark}\label{rmk:cyclic-structure}
    \todo{This remark is not necessary}
    Merk at for en syklisk undergruppe $\langle g\rangle\subset G$
    må vi ihvertfall ha $e = g^0, g^{-1}\in \langle g\rangle$,
    og $gg = g^2\in \langle g\rangle$,
    så ved induksjon følger det at $g^n\in \langle g\rangle$
    for alle $n\in \mathbb Z$.
\end{remark}

\begin{example}
    Heltallene $\mathbb Z$ er en syklisk gruppe generert av $1$.
    Alle undergruppene $n\mathbb Z\subset \mathbb Z$ er også sykliske generert av
    $n$.
\end{example}

\begin{lemma}
    La $G$ være en gruppe.
    Da er $G$ syklisk hvis og bare hvis det finnes en surjektiv morfi
    $\mathbb Z\to G$.
\end{lemma}
\begin{proof}
    Anta det finnes en surjektiv avbildning $f\colon \mathbb Z\to G$.
    Da har vi at for alle $g\in G$ så finnes en $n\in \mathbb Z$ slik at
    $f(n) = f(1)^n = g$, så $G = \langle f(1)\rangle$.

    Anta at $G = \langle g \rangle$ for en $g\in G$.
    Vi konstruerer en avbildning $f\colon \mathbb Z\to G$
    gitt ved $n\mapsto g^n$.
    Åpenbart har vi at $\im f\subset G$,
    men $\im f$ er en undergruppe av $G$ og $g\in \im f$,
    så $\langle g\rangle\subset \im f$ siden $\langle g\rangle$
    er den minste undergruppen som inneholder $g$.
    Men $\langle g\rangle = G$,
    så dermed har vi at $\langle g \rangle = \im f = G$,
    så $f$ er surjektiv.
\end{proof}

\begin{corollary}
    Undergruppene $k\mathbb Z = \langle k\rangle \subset \mathbb Z$
    er de eneste undergruppene $\mathbb Z$ har.
\end{corollary}
\begin{proof}
    La $k_1, \dots, k_n$ være en endelig samling elementer i $\mathbb Z$.
    Da finnes det heltall $c_1,\dots,c_n$ slik at $c_1 k_1 + \dots + c_n k_n = k$
    hvor $k = \mathrm{gcd}(k_1,\dots,k_n)$
    er største felles nevner for $k_1,\dots, k_n$.
    Dette følger fra Euclids algoritme.
    Da vil $\langle k_1,\dots, k_n\rangle = \langle k\rangle$.

    Et tilsvarende argument holder for uendelige samlinger $\{k_i\}$
    ved å velge en endelig delmengde.
\end{proof}

\begin{corollary}
    La $G$ være en syklisk gruppe.
    Da finnes en isomorfi $\mathbb Z / n\mathbb Z\to G$
    hvor $n = \# G$.
\end{corollary}
\begin{proof}
    \todo[inline]{Bevis dette}
\end{proof}

\begin{definition}
    La $G$ være en gruppe.
    En undergruppe $H\subset G$ er \textit{ekte}
    om $H\neq \{e\}$ og $H\neq G$.

    En gruppe $G$ er \textit{simpel} om $G$ ikke har noen ekte undergrupper.
\end{definition}

\begin{theorem}
    Om $A$ er en endelig abelsk gruppe, og $\# A$ er et primtall,
    da er $A$ simpel.
\end{theorem}

\begin{lemma}\label{thm:binomial-identity}
    La $x,y\in \mathbb Z$  og $n \geq 1$ være heltall.
    Da har vi formelen
    \[
        (x + y)^n = \sum_{i = 0}^n \binom n i x^i y^{n - i}.
    \]
\end{lemma}

Av og til tar vi dette som definisjonen av \textit{binomialkoeffisientene}
$\binom n m$,
men om vi heller bruker definisjonen fra Pascals trekant $\binom n 0 = \binom n n = 1$
for alle $n\in\mathbb N$ og $\binom n {m - 1} + \binom n m = \binom {n + 1} m$
for alle $n, m$ kan vi vise \cref{thm:binomial-identity} ved induksjon som følger.
\begin{proof}
    Tilfellet $n = 1$ er enkelt siden
    $(x + y)^1 = x + y$ og $\binom 1 0 = \binom 1 1 = 1$.

    Anta $(x + y)^k = \sum_{i = 0}^k \binom k i x^i y^{k - i}$.
    Vi har
    \[\begin{aligned}
        (x + y)^{k + 1}
        &=  (x + y)(x + y)^k
        \\
        &= (x + y)\sum_{i = 0}^k \binom k i x^i y^{k - i}
        \\
        &= \sum_{i = 0}^k \binom k i x^{i + 1} y^{k - i}
        + \sum_{i = 0}^k \binom k i x^{i} y^{k - i + 1}
        \\
        &= \sum_{i = 1}^{k + 1} \binom k {i - 1} x^{i} y^{(k + 1) - i}
        + \sum_{i = 0}^k \binom k i x^{i} y^{(k + 1) - i}
        \\
        &= \sum_{i = 0}^{k + 1} \binom {k + 1} i x^{i} y^{(k + 1) - i}
    \end{aligned}\]
\end{proof}

\begin{corollary}{Fermats lille teorem}
    La $p$ være et primtall.
    Da har vi at
    \[
        a^p \cong a \mod p
    \]
    for alle $a\in \mathbb Z$.
\end{corollary}
\begin{proof}
    Vi definerer en avbildning $\pi_p\colon \mathbb Z / p\to\mathbb Z / p$
    ved $a\mapsto a^p$.
    La $a,b\in \mathbb Z$.
    Vi ser at
    \[\begin{aligned}
        \pi_p(a + b)
        &= (a + b)^p
        \\
        &= \sum_{i = 0}^p \binom p i a^i b^{p - i}
        \\
        &= a^p + b^p
        \\
        &= \pi_p(a) + \pi_p(b)
    \end{aligned}\]
    siden $p|\binom p i$ for $i\neq 0, p$,
    så $\pi_p$ er en morfi.
    Vi ser også at $\pi_p(1) = 1^p = 1$,
    og det er bare \`en morfi $\mathbb Z / p\to \mathbb Z / p$
    som tilfredsstiller dette, nemlig identiteten $\id_{\mathbb Z / p}$,
    så $\pi_p = \id_{\mathbb Z / p}$.
\end{proof}

\subsubsection*{Oppgaver}
\begin{enumerate}
    \item La $A$ være en abelsk gruppe og $H\subset A$ en undergruppe.
        \begin{enumerate}
            \item Vis at $H$ er abelsk.
            \item Vis at $H$ er en normal undergruppe.
        \end{enumerate}
\end{enumerate}
